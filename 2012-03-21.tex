\documentclass{school-22.211-notes}
\date{March 21, 2012}

\begin{document}
\maketitle

\lecture{Multi-Group Diffusion: Derivations}
This class would not require derivation of the transport theory. 
\topic{Multi-Group Diffusion Theory}

The diffusion coefficient approximation table is very important: $\frac{1}{3\Sigma_t}$ is off by a factor of 3! 



The group diffusion coefficient $D_g$ should be a tensor; but in application, we assume that $D_g$ is the same for all directions. A more accurate approach is to let $D_g$ be the same in the x-y plane, but different in axial direction. 


\clearpage
\topic{Partial Currents}
\begin{enumerate}
\item Scalar flux, 
  \eqn{\phi(\vecr, E) = \int_{4\pi} \psi(\vecr, E, \vecOmega) \dOmega }
\item Net current,
  \eqn{\vecJ(\vecr, E) = \int_{4\pi} \vecOmega \psi(\vecr, E, \vecOmega) \dOmega  }
\item Partial Currents in diffusion theory:
  \begin{align}
    J^+(\vecr, E) &= \int_{\vecn \cdot \vecOmega > 0} \vecn \cdot \vecOmega \psi(\vecr, E, \vecOmega) = \frac{1}{4\pi} \int_{\vecn \cdot \vecOmega > 0} \vecn \cdot \vecOmega [\phi(\vecr, E) + 3 \vecOmega \cdot \vecJ(\vecr, E) ] \\
    &= \frac{1}{4} \phi(\vecr, E) + \frac{1}{2} J_n (\vecr, E) \\
    J^-(\vecr, E) &= \int_{\vecn \cdot \vecOmega < 0} |\vecn \cdot \vecOmega| \psi(\vecr, E, \vecOmega)  \\
    &= \frac{1}{4} \phi(\vecr, E) - \frac{1}{2} J_n (\vecr, E) 
  \end{align}
\end{enumerate}


\clearpage
\topic{Eigenvalue Problem}
We start from a one-group diffusion equation,
\eqn{ - \divergence D(\vecr) \gradient \phi(\vecr) + \Sigma(\vecr) \phi(\vecr) = \frac{1}{\keff} \nu \Sigma_f(\vecr) \phi(\vecr) }
Assume cross sections do not depend on spatial distribution, 

Rearranging, 

Define \hi{material buckling} 
\eqn{ B^2_m = }
Then we get our \hi{Helmholtz equation}:


For a criticality problem, we can solve find the $B^2$ such that the system is critical. Hence giving geometry, only certain materials would make the system critical; given material, only certain geometries would make the system critical. That is, given a reactor, it would be critical if and only if the material and the geometry satisfy,
\eqn{ B_g^2 = B_m^2} 
and there can only be a solution when $\kinf > 1$ (exception: if there is external source, $\kinf$ can be less than 1 and the system is still critical). 


$DB^2_g$ is the leakage per unit volume per unit flux. Notice $\keff$ does not depend on volume or flux. 


\hi{migration area} $M^2 = \frac{D}{\Sigma_a}$ is a measurement of the amount of travelling before absorption. 


\end{document}
