\documentclass{school-22.211-notes}
\date{February 15, 2012}

\begin{document}
\maketitle

\lecture{Slowing Down with Resonance Absorption}
At the end of last lecture, we get a smooth flux vs. lethargy curve, with two peaks (the one in the higher energy is stronger), and a smooth curve inbetween. Now we are going to look at the middle resonance region. 

\topic{Setting Up A Resonance Model}
\subtopic{Model Assumptions}
For our resonance model, we use the Single Level Breit-Wigner(SLBW) and assume:
\begin{itemize}
  \item the only resonance isotope is U238;
  \item all resonances are well isolated;
  \item only s-wave interaction (scattering \& absorption);
  \item Reich-Moore parameters can be used in SLBW;
  \item treat as resolved only the lowest 14 s-wave resonances;
  \item generate simple `statistical model' for energies up to 10 keV, assuming an uniform spacing 25 eV\footnote{notice on a typical plot which is log-log, the spacing appears to be closer and closer as energy increases; though it is actually reasonable to assume that the spacing is uniform};
\end{itemize}

\subtopic{Resonance Data}
One place to get resonance data is from LANL's website. For instance, \href{http://t2.lanl.gov/cgi-bin/endf?2,151,/inet/WWW/data/data/ENDFB-VII-neutron/U/238}{U238 Resonance Parameters}. Among the tables, GN means $\Gamma_N$ with unit eV, and GG means $\Gamma_G$ in eV. Extract the energy range from 0 to 10 keV (ignore the negative resonance energies). 

\topic{Model For Resolved Resonance Absorption}
\begin{align}
\sigma_{\gamma} (E,T) &= \sqrt{\frac{E_0}{E}} \frac{2}{\Gamma} A \psi(x,\xi) \\
\sigma_{n} (E,T) &= \frac{2}{\Gamma} \left[ A \psi(x,\xi) + B \chi(x,\xi) \right] + \sigma_{\mathrm{potential}} 
\end{align}
To convert them into a form suitable for numerical calculation,
\begin{align}
\sigma_{\gamma} (E,T) &= \sqrt{\frac{E_0}{E}} \frac{\Gamma_n}{\Gamma} \frac{\Gamma_{\gamma}}{\Gamma} r \psi(x,\xi) \\
\sigma_{n} (E,T) &= \frac{\Gamma_n}{\Gamma} \frac{\Gamma_n}{\Gamma} \left[ r \psi(x,\xi) + q \chi(x,\xi) \right] + \sigma_{\mathrm{potential}} 
\end{align}
where
\begin{align}
\xi &= \Gamma \sqrt{\frac{A}{4 k T E_0}} \\
x &= \frac{2 (E-E_0)}{\Gamma} \\
r &= \frac{h^2}{2 \pi E_0} \frac{A+1}{A} &= \frac{2603911}{E_0} \frac{A+1}{A} \\
q &= \sqrt{ r \sigma_{\mathrm{potential}} } \\
\Gamma &= \Gamma_n + \Gamma_{\gamma} \\
\sigma_{\mathrm{potential}} &=  4 \pi R^2 
\end{align}
Notice there are many numerical representations of the $\psi, \chi$ functions. For now, we can use ....

Results: It is important to notice that \textit{all resonances contribute to all energies, even though their contributions can be infinitesimally}. 


\subtopic{Model For Unresolved Resonance Absorption}
We assume 25 eV resonance spacing, and assume
\begin{align}
\Gamma_{\gamma} &= 0.023 \eV \\
\Gamma_{n} = 0.05 \sqrt{\frac{E}{E_{\mathrm{last}}}}  \eV
\end{align}
in which $E_{\mathrm{last}}$ means the last energy used in the 14 resonance region. 

Notice that for the purpose of this excercise we assume this region to be unresolved, whereas in reality they are resolved. 


\end{document}
