\documentclass{school-22.211-notes}
\date{March 14, 2012}

\begin{document}
\maketitle

\lecture{Intro to Diffusion Theory}
\topic{Why We Need Deterministic Codes}
Lec10. Monte Carlo requires too much computing power. 


Pin-cell based, 


Assumbly based. 




In the next section we will discuss how to solve diffusion equation. 


\topic{k-infinity In Infinite Medium Calculations (No Leakage)}
Reference: Henry p. 104, Lec10 [FIXME] 
\eqn{ k_{\infty} =  \frac{\mbox{total fissions} \cdot \bar{\nu} }{\mbox{total absorptions}} = \frac{\int \dV \int \nu \overline{\Sigma_f} \Phi \dE}{\int \dV \int \overline{\Sigma_a} \Phi \dE} = \epsilon \eta f p }

\subtopic{k from one-group cross section}




\subtopic{k from two-group cross section}
\begin{enumerate}
\item General case: 
  \begin{align}
    \frac{\nu \overline{\Sigma_{f1}}}{\kinf} - \overline{\Sigma_{a1}} \phi_1 - \\
  \end{align}

\item Using effective removal cross section. We define an \hi{effective removal cross section} as,
  

   Although it looks like a chicken-to-egg problem that we include $\frac{\phi_2}{\phi_1}$ in $\hat{\Sigma}_{s12}$, my understanding is that we either know flux and needs to find cross section, or the other way around, hence we do not need to worry about this term. 

   If no upscattering, $\hat{\Sigma}_{s12}$ is effectively $\overline{\Sigma}_{s12}$, hence 
\end{enumerate}

\subtopic{k From Balance Equation}
For two region problem, we can also solve $\kinf$ from the neutron balance equation. 





The $\kinf$ from the three methods above should agree, because reaction rates conserves[FIXME].




\end{document}
