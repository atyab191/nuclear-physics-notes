\documentclass{school-22.211-notes}
\date{May 23, 2012}

\begin{document}
\maketitle

%%%%%%%%%%%%%%%%%%%%%%%%% Final Exam Start %%%%%%%%%%%%%%%%%%%%%%%%%%%%
\lecture{Exam 2 Review}
\topic{Point Kinetics}
\begin{enumerate}
\item Basic Concepts: 
  \begin{itemize}
  \item Burst measurement: prompt neutrons; saturated: total neutrons. 
  \item Dominant FP that emits delayed neutrons: Br-37. 
  \item Delayed yields depend on: fissioning species, neutron energy.  
  \item Absolute yield: number of delayed neutrons per fission; unit: neutrons/fission. Eg: U238 = 4\%, U235=1.7\%. 
  \item Relative yield: absolute yield of an isotope divided by total absolute yield by all isotopes; unit: percentage. 
  \item Delayed neutron fraction: absolute yield divided by $\bar{\nu}$. Number of delayed neutrons devided by number of total fission neutrons. 
  \item Energy aspect of delayed neutron spectra: average prompt neutron emission energy: 2 MeV for prompt neutrons, 0.4 MeV for delayed neutrons. Both spectrum are Maxwellian, except delayed neutrons are more thermal, which makes them more likely to fission. 
  \item Energy group aspect: Delayed neutron spectra vary very slightly for different fissioning nuclides, but very significantly on energy groups. 
  \end{itemize}
\end{enumerate}


\clearpage
\topic{Homogenization Methods}
\begin{enumerate}
\item
\end{enumerate}


\clearpage
\topic{De-homogenization Methods}
\begin{enumerate}
\item
\end{enumerate}


\clearpage
\topic{Adjoint Fluxes}
\begin{enumerate}
\item
\end{enumerate}
%%%%%%%%%%%%%%%%%%%%%%%%% Final Exam End %%%%%%%%%%%%%%%%%%%%%%%%%%%%




\end{document}
