\documentclass{school-22.211-notes}
\date{March  5, 2012}

\begin{document}
\maketitle

\topic{Heterogeneous Geometry Resonant Approximations}
This is probably the most important topic in generating cross section in reactor physics. We want to take into account both the spatial and energy structure distributions of flux within each resonance when solving the neutron slowing down problem. 

Assumptions for heterogeneous geometry resonant approximations:
\begin{itemize}
\item All neutron interactions in moderator are purely scattering;
\item Moderator scattering xs is independent of energy;
\item The previous two points together leads to that the slowing down energy source in fuel and moderator is 1/E;
\item Slowing down source is spatially uniform within each region (a fairly accurate assumption);
\item A single resonance absorber species exists only in the fuel;
\item Fuel scattering removes neutrons from resonance energy (that is, narrow resonance model); 
\end{itemize}

\subtopic{Spatial Reciprocity Theorem}
First we introduce the reciprocity condition. The only assumption we make for this theorem is that the source is flat. This theorem does not depend on geometry. 

Given two homogeneous region 1 and 2, the probability of going from point A in 1 to point B in 2 without colliding is:
\eqn{ \frac{e^{-\tau}}{4 \pi R^2} }
where $\tau = \frac{x_1}{\lambda_1} + \frac{x_2}{\lambda_2} = x_1 \Sigma_1 + x_2 \Sigma_2$. The probability of going from A to B and then colliding at B is: 
\eqn{ \frac{e^{-\tau}}{4 \pi R^2} \Sigma_2 }
The probability of going from A to any point in region 2 and colliding is:
\eqn{ \int \frac{e^{-\tau}}{4 \pi R^2} \Sigma_2  \dV_2}
The probability of a uniform spatial neutron source in region 1 going to any point in region 2 and colliding is:
\eqn{ \int \dV_1 \int \frac{e^{-\tau}}{4 \pi R^2} \Sigma_2  \dV_2}
The average probability per unit volume of a uniform source neutron going from region 1 to 2 and colliding is:
\eqn{ P_{1\to 2} = \frac{\Sigma_2}{V_1} \int \dV_1 \int \frac{e^{-\tau}}{4 \pi R^2}  \dV_2}
Likewise,
\eqn{ P_{2\to 1} = \frac{\Sigma_1}{V_2} \int \dV_2 \int \frac{e^{-\tau}}{4 \pi R^2}  \dV_1}
Since the two integrals are symmetric (that is, $P_{A\to B}$ and $P_{B \to A}$ are essentially the same), we can write,
\eqn{ P_{1\to 2} \frac{V_1}{\Sigma_2} = P_{2\to 1} \frac{V_2}{\Sigma_1} }
Hence we reach the reciprocity condition:
\eqn{ \boxed{ P_{1\to 2} \Sigma_1 V_1 = P_{2\to 1} \Sigma_2 V_2 } }

\subtopic{First Collision Reaction Rate Balance}



\subtopic{Use Reciprocity to Construct 2-Region Balance Equations}


$\sigma_{t,f}(u) = \sigma_{r,f} + \sigma_{pot, f}$, the resonance absorption xs plus the potential scattering xs. 


\subtopic{Heterogeneous/Homogeneous Equivalence}
\hi{Heterogeneous/Homogeneous Equivalence} says that we can compare the heterogeneous energy shape of the flux in the fuel: 


\subtopic{Bell's Refinements of the Wigner's Collision Probability}
Bell Factor $b$:
\eqn{ P_{f\to f} = \frac{\sigma_{t,f} (u)}{b \sigma_e + \sigma_{t,f} (u)} }


\subtopic{Arrays of Rods, Dancoff Factor}
Notice our model so far is an isolated pin (that is, a fuel pin surrounded by an almost infinite moderator). Now we are going to discuss arrays of rods. We define the \hi{Dancoff Factor C}:
\eqn{ C = 1 - \frac{ \left. P_{f\to f} \right|_{\mbox{isolated rods}}}{ \left. P_{f\to f} \right|_{\mbox{arrays of rods}}}   }
LWRs typically have a C of 0.3. Know how to get C from the plot. 


C changes the coefficient in front of the escape xs $\sigma_e \to \frac{(1-C)b}{1-C + Cb} \sigma_e$. For a typical PWR pin, that is reduce $\sigma_e$ to 0.745 of it. \ce{^{235}U}'s xs is 65 barns [FIXME].

\uline{Example: Find Dancoff Factor} Given water at 1 g/cc with a number density of $6.6 \times 10^{24}$ H atoms/cc, hydrogen xs of 20 barns, a PWR pin with 0.41 cm radius, a lattice pitch of 1.26cm. 

Answer: we first find the mean free path: 
\eqn{ mfp = \frac{1}{\Sigma} = \frac{1}{N \sigma} =  }



\subtopic{Carlvik's Refinements of the Bell's Collision Probability}
Carlvik's two-term approximation is what is actually used in production tools nowaday. To use it, we essentially look up the resonance integral (or group xs) twice, once with $\sigma_e$ multiplied by $\alpha_1$ and the second time with $\sigma_e$ multiplied by $\alpha_2$, each of them are simple functions of the Dancoff factor. To double check, we let Dancoff factor to go to 1, and there should be no escape from the fuel; let Dancoff factor go to 0, we should get isolated rod. 



\end{document}
