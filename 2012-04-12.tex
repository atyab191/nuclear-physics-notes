\documentclass{school-22.211-notes}
\date{April 12, 2012}

\begin{document}
\maketitle

\lecture{Exam 2 Review}
\topic{General Background}
\begin{enumerate}

\item Material $\kinf$ from group cross sections: 
  \begin{itemize}
  \item One-group: no flux dependency; cross sections that treat pin-cells as homogenized preserve exactly our MC fuel reactivity for infinite repeating lattices. 
  \eqn{ \kinf = \frac{\nu \bar{\Sigma}_f }{\bar{\Sigma}_a} }
  \item Two-group: Solve from two-group balance equation, still only depend on integrated cross section (we approximate the flux ratio with cross sections). 
    \eqn{ \kinf = \frac{\nu \bar{\Sigma}_{f1} + \nu \bar{\Sigma}_{f2} \frac{\hat{\Sigma}_{s12} }{\bar{\Sigma}_{a2}}}{\bar{\Sigma}_{a1} + \hat{\Sigma}_{s12} }   }
  \item Balance equation, \textcolor{red}{what is it?} 
  \end{itemize}

\item $\keff$: take into account leakage,
  \eqn{ \keff = \frac{\nu \Sigma_f}{DB^2 + \Sigma_a} } 
  Notice $\keff$ does not depend on volume or flux. 

\item Materials bucklings,
\eqn{ B_m^2 = \frac{\frac{\nu \Sigma_f}{\keff} - \Sigma_a}{D} }

\item Geometrical buckling: the allowable values that satisfy the boundary conditions are uniquely determined reactor geometry. 


\item Infinite medium critical buckling:
  \eqn{ B_m^2 = \frac{\nu \Sigma_f - \Sigma_a}{D} = \frac{\frac{\nu \Sigma_f}{\Sigma_a} - 1 }{\frac{D}{\Sigma_a}} = \frac{\kinf - 1}{M^2} }
  where migration area $M^2$ is a measurement of the distance travelled before absorption. 

\item Eigenvalues/eigenfunctions
\item Fission yields
\item Transient fission product equations
\item I/Xe reactivity effects
\item Pm/Sm reactivity effects
\item Divergence theorem
\item Laplacian 
\begin{table}[ht]
  \centering
  \begin{tabular}{|l|l|} \hline
    Slab & $\dphidxn2 + B^2 \phi(x) = 0$ \\ \hline
    Sphere & $\dphidrn2 + \frac{2}{r} \dphidr + B^2 \phi(r) = 0$ \\ \hline
    Infinite Cylinder & $\dphidrn2 + \frac{1}{r} \dphidr + B^2 \phi(r) = 0$ \\ \hline
    Finite Cylinder & $\dphidrn2 + \frac{1}{r} \dphidr + \dphidzn2 + B^2 \phi(r,z) = 0$ \\ \hline
    Cartesian & $\dphidxn2 + \dphidyn2 + \dphidzn2 + B^2 \phi(x,y,z) = 0$ \\ \hline
  \end{tabular}
\end{table}

\end{enumerate}



\clearpage
\topic{Analytical Diffusion Theory} 
\begin{enumerate}
\item Transport cross section and diffusion coefficients: from the net current equation, assume the scattering is isotropic in the COM system, and use transport correction $P_0$ approximation, 
  \eqn{ D = \frac{1}{3 \Sigma_{tr} } }
  \eqn{ \Sigma_{tr} = \Sigma_t - \Sigma_{s1} = \Sigma_t - \frac{2}{3A} \Sigma_s }

\item Effective down-scattering cross section: 
  \eqn{ \hat{\Sigma}_{s12} = \bar{\Sigma}_{s12} - \bar{\Sigma}_{s21} \frac{\phi_2}{\phi_1} }
  So that up-scattering is zero: 
  \eqn{ \Sigma_{21}  = 0 }

\item Removal cross section
  \eqn{ \Sigma_{rg} = \Sigma_{tg} - \Sigma_{sgg} = \Sigma_{ag} + \Sum_{g'=1, g'\neq g}^G \Sigma_{sgg'}  }

\item Two-group diffusion equations,
  \begin{align}
    \left[ \begin{array}{cc} 
        \frac{\nu \bar{\Sigma}_{f1}}{\kinf} -  \bar{\Sigma}_{a1}  - \hat{\Sigma}_{s12} & \frac{\nu \bar{\Sigma}_{f2}}{\kinf}   \\
        \hat{\Sigma}_{s12} &  - \bar{\Sigma}_{a2}  
      \end{array} \right] 
    \left[ \begin{array}{c} \phi_1 \\ \phi_2 \end{array} \right] = 0
  \end{align}

\item Multi-group diffusion equations: See Section~\ref{multi-group-diffusion}. 
\begin{align}
& - \overbrace{\int_{E_g}^{E_{g-1}} \dE \divergence D(\vecr, E) \gradient \phi(\vecr, E)}^{\mbox{leakage/diffusion term }\textcircled{1}} + 
\overbrace{\int_{E_g}^{E_{g-1}} \dE \Sigma_t(\vecr, E) \phi(\vecr, E)}^{\mbox{total interaction term }\textcircled{2}} = \overbrace{\int_{E_g}^{E_{g-1}} \dE S(\vecr, E)}^{\mbox{source term } \textcircled{3}}  \\
& + \overbrace{\int_{E_g}^{E_{g-1}} \dE \chi(E) \int_{E'} \dE' \nu \Sigma_f (\vecr, E') \phi(\vecr, E')}^{\mbox{fission source term }\textcircled{4}} 
 + \overbrace{\int_{E_g}^{E_{g-1}} \dE \int_{E'} \dE' \Sigma_s(\vecr, E'\to E) \phi(\vecr, E')}^{\mbox{scattering source term }\textcircled{5}} 
\end{align}
\eqn{ -\divergence D_g(\vecr) \gradient \phi_g (\vecr) + \Sigma_{tg} (\vecr) \phi_g(\vecr) = \chi_g\Sum_{g'=1}^G \nu \Sigma_{fg'} \phi_{g'}(\vecr) + \Sum_{g'=1}^G \Sigma_{sg'g} (\vecr) \phi_{g'} (\vecr) + S_g(\vecr) }
Cancelling the within group scattering cross section from both sides and defining the group-wise removal cross section, we get the final form,
\eqn{ \boxed{- \divergence D_g(\vecr) \gradient \phi_g(\vecr) + \Sigma_{rg} (\vecr) \phi_g(\vecr) = \chi_g \Sum_{g'=1}^G \nu \Sigma_{fg'} (\vecr) \phi_{g'} (\vecr) + \Sum_{g'=1,g'\neq g}^G \Sigma_{sg'g} (\vecr) \phi_{g'} (\vecr) + S_g(\vecr) } }

\item Partial currents and albedos:
  \begin{align}
    J^+(\vecr, E) &= \int_{\vecn \cdot \vecOmega > 0} \vecn \cdot \vecOmega \psi(\vecr, E, \vecOmega) = \frac{1}{4\pi} \int_{\vecn \cdot \vecOmega > 0} \vecn \cdot \vecOmega [\phi(\vecr, E) + 3 \vecOmega \cdot \vecJ(\vecr, E) ] \\
    &= \frac{1}{4} \phi(\vecr, E) + \frac{1}{2} J_n (\vecr, E) \\
    J^-(\vecr, E) &= \int_{\vecn \cdot \vecOmega < 0} |\vecn \cdot \vecOmega| \psi(\vecr, E, \vecOmega)  \\
    &= \frac{1}{4} \phi(\vecr, E) - \frac{1}{2} J_n (\vecr, E) 
  \end{align}
  Albedo boundary condition is the measurement of how much flux is reflected back: 
  \eqn{ \alpha = \frac{J^- (\vecr_i, E)}{J^+ (\vecr_i, E)} }
  
\item Boundary conditions:
  \begin{enumerate}
  \item Zero flux boundary condition:
    \eqn{ \phi (0) = 0 }

  \item Zero incoming flux bc, integrating over all angles in half space, it is equivalent to zero incoming partial current:
    \eqn{ \left. \psi(\vecr, E, \vecOmega) \right|_{\vecn \cdot \vecOmega < 0} &= 0, & J^- (\vecr_i, E) &= 0}
    There are two formulism to solve this:
    \begin{itemize}
    \item Kord's formulism: 
      \eqn{ J^- (\vecr, E) &= \frac{1}{4} \phi(\vecr, E) - \frac{1}{2} J_n (\vecr, E)  = 0, & \Aboxed{ \frac{J}{\phi} &= \frac{1}{2} } }
    \item The more conventional approach is to approximate with extropolation boundary condition: 
      \begin{align}
        J^- &= \frac{1}{4} \phi + \frac{D}{2} \gradient \phi_n \\
        \Aboxed{ \frac{\gradient \phi}{\phi} &= - \frac{1}{2D} = - \frac{3\Sigma_{tr}}{2} = - \frac{1}{d_{\mathrm{extrap}}}} 
        \mbox{   where } d_{\mathrm{extrap}} = \frac{2}{3 \Sigma_{tr}}  = \frac{2}{3} \lambda_{\mathrm{tr}}
      \end{align}
      where $D$ is the property of the material inside, and $\lambda_{tr}$ is transport mean free path. The coefficient before $\lambda_{tr}$ can be 0.711 in other formulism. 
    \end{itemize}
  \end{enumerate}

\item Interface conditions: 
  \begin{enumerate}
    \item Continuity of scalar flux: 
      \eqn{ \phi(\vecr_i^-, E) = \phi(\vecr_i^+, E) }
    \item Continuity of normal current:
      \eqn{ \vecn \cdot D(\vecr_i^-, E) \gradient \phi(\vecr_i^-, E) = \vecn \cdot D(\vecr_i^+, E) \gradient \phi(\vecr_i^+, E) }
  \end{enumerate}
\end{enumerate}

\clearpage
\topic{Other Diffusion-Related Equations}
\begin{enumerate}
\item Balance equation derived from transport equations,
  \eqn{ \divergence \vecJ (\vecr, E) + \Sigma_t (\vecr, E) \phi(\vecr, E) = \Sum_i \int_{E'} \dE' \Sigma_s^i (\vecr, E'\to E) \phi(\vecr, E') + S_0 (\vecr, E) }
This expression is exact, because we haven't approximate $\vecJ$ yet. 

\item Using $P_1$ angular flux expansion,
  \eqn{ \psi (\vecr, E, \Omegahat) = \frac{1}{4\pi} \left[ \phi(\vecr, E) + 3 \Omegahat \cdot \vecJ(\vecr, E) \right] }
  We can get the net current equation,
  \eqn{ \frac{1}{3} \gradient \phi(\vecr, E) + \Sigma_t (\vecr, E) \vecJ(\vecr, E) = \Sum_i \int_{E'} \dE' \Sigma_{s1}^i (\vecr, E'\to E) \vecJ(\vecr, E') + \vec{S}_1(\vecr, E) }

\item From net current equation, assume no source and isotropic scattering in lab system, we can approximate the transport equation as, 
  \eqn{ \Sigma_{tr}^i (E) = \Sigma_t^i (E) - \frac{2}{3A^i} \Sigma_s^i (E) }
  Then we can solve for diffusion coefficient $D = \frac{1}{3 \Sigma_{tr}}$. 

\item One-group diffusion equation,
  \eqn{ -\divergence D(\vecr) \gradient \phi(\vecr) + \Sigma_a(\vecr) \phi(\vecr) = \frac{1}{\keff} \nu \Sigma_f(\vecr) \phi(\vecr) }

\item Helmoltz Equation: from one-group diffusion equation, we assume spatial constant cross section and use buckling term, 
  \eqn{ \laplacian \phi(\vecr) + B^2 \phi(\vecr) = 0 }
  This equation is important in showing that $\phi(\vecr)$ has a constant curvature. 
\end{enumerate}

\clearpage
\topic{Practise Problems}
\begin{enumerate}
\item Homogeneous (single-region) criticality problems:
  \begin{enumerate}
  \item 1D slab $\in  \left[- \frac{L_0}{2}, \frac{L_0}{2} \right]$, no source, critical. 
    \eqn{ \dphidxn2 + B^2 \phi (x) &= 0, &\phi(x) &= A \cos (Bx) + C \sin(Bx) }
    BCs: $\phi(\pm L/2) = 0$, where $\frac{L}{2} = \frac{L_0}{2} + 0.711 \lambda_{tr}$. Two equations two unknowns, 
    \eqn{ \left[ \begin{array}{cc} \cos(BL/2) & \sin(BL/2) \\ \cos(BL/2) & -\sin(BL/2) \end{array} \right] \left[ \begin{array}{cc} A \\ C \end{array} \right] = 0 }
    Set the determinant to be zero, we get $-2 \cos (BL/2) \sin (BL/2) = 0$. There are two possibilities: 
    \eqn{ B_n&= \frac{n\pi}{L}  & \phi(x) &= \left\{ 
      \begin{array}{cc} 
        A_n \cos (B_n x) & n=1,3,5, \cdots \\
        A_n \sin (B_n x) & n=2,4,6, \cdots 
      \end{array} \right. }
    But in order for $\phi(x) \ge 0$ everywhere, only $n=1$ is possible; that is, 
    \eqn{ \phi(x) = A \cos \frac{\pi x}{L} }
    and the criticality condition implies that, 
    \eqn{ \frac{\nu \Sigma_f - \Sigma_a}{D}  = \left( \frac{\pi}{L} \right)^2 }

  \item Find the height-to-diameter that would minimize the leakage from a fixed-volume cylinder. To minimize leakage, we need to maximize $B_m^2$ hence $B_g^2$. For a finite cylinder, we know
    \eqn{B_g^2 = \left( \frac{2.405}{R} \right)^2 + \left( \frac{\pi}{H} \right)^2 }
    and we solve $\frac{\derivative B_g^2}{\dr} = 0$. The optimal H/D ratio comes out to be 0.9236. 
    
  \end{enumerate}
  Summary: only the lowest node remains after the source is gone. For any positive value of material buckling, there is a unique critical size for each geometry. 
  
\item Homogeneous (single-region) source problems:
  \begin{enumerate}
  \item See Section~\ref{one-group-source-problem-subcritical}
  \item 
  \end{enumerate}
  
\item Multi-region diffusion problems
\end{enumerate}

\clearpage
\topic{Numerical Diffusion Theory}
\begin{enumerate}
\item Finite-difference diffusion equation
\item Fission source iterations
\item Gauss-Jacobi flux inner iterations
\item Dominance ratio estimation
\end{enumerate}


\end{document}
