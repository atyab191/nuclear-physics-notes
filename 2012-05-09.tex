\documentclass{school-22.211-notes}
\date{May  9, 2012}

\begin{document}
\maketitle

\lecture{Reconstruction/De-homogenization Methods}
\topic{Superposition Assumptions}
Assembly-wise cross sections, pin-wise cross sections with spacial dependency. 




\topic{Non-Separable Flux Expansion}
\begin{enumerate}
\item Using 5th order polynomial, there are 25 unknown coefficients in the fast group (5 for each of the two directions). 
\item There appears to be 50 unknowns in the thermal groups; but it is really 25, because the other 25 for the specific colution can be solved from the fast group.
\item Surface constrains preserve flux shape. 
\textcolor{red}{each node has 4 surface constraints?} 
\end{enumerate}


\topic{Reconstruction Steps}



\topic{Applications}
To measure power of a core, we pull out individual pins and measure the gamma deposited in it to get the power profile of it. The uncertainty comes out to be about 1.5\%, which is considered to be high precision. 


\topic{Fixing Cross-sections On The Fly/`Adhop' Method}
SA: single assembly. 

This method is somewhat an empiracle one; hence it is not very robust. Nowadays, nodal methods just go to higher order of energy groups. 

\clearpage
\topic{Nodal Method Summary}
Basically a production-grade nodal code must have, 
\begin{itemize}
\item Accurate lattice data;
\item Coupling to 
\end{itemize}


\lecture{Adjoint Fluxes}
\topic{Review}

We can get a much accurate prediction of reactivity without knowning $\delta \psi$. 



There is no reactivity associated with a space and a time (it has a reactivity contribution). Reactivity is always integrated over the entier system because of the denominator. 



Because of the linearility of the first-order perturbation theory, we can super-impose: 
\begin{itemize}
\item reactivity contributions from perturbations in different spatial regions;
\item reactivity effects from perturbations in xxx. 
\end{itemize}


\hi{The adjoint flux is defined as the asymptotic increase in total neutron population of a critical reactor for a neutron introduced a position r, direction omega, and energy E.} The adjoint is only defined for what reaction that we are interested in: in a critical reactor, that is neutron population; in a subcritical system, like the one in detector, then it is whatever purpose, like detector response, that we are interested in. 






\end{document}
