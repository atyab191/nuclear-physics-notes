\documentclass{school-22.211-notes}
\date{February  8, 2012}

\begin{document}
\maketitle

\lecture{Introduction}
\topic{Course Objectives}
\begin{itemize}
  \item Continuous energy transport, reduction to multi-group diffusion group;
  \item Resonance absorption and spatial self-shielding models;
  \item Calculation of neutron spectra;
  \item Determination of few-group diffusion constants;
  \item Elementary reactor transient analysis.
\end{itemize}

\topic{Exercises}
\subtopic{Estimate the Mean Free Path of a Fission Neutron in an LWR} 


\lecture{Neutron Slowing Down}
Next time: elastic scattering; xi spectrum; calculate the flux for different reactor types. 
\topic{Preview: Reuss Ch7}
\begin{enumerate}
\item Decouple absorption and scattering is possible because absorption is complicated at lower energies, whereas scattering is the opposite (b/c inelastic and anisotropic aspects). 

\item Elastic vs inelastic: elastic scattering has no threshold, making it the most important one in neutron slowing down. 

\item Laws of elastic collision: 
  \begin{align}
    \frac{E_{nf}}{E_{ni}} &= \frac{A^2 + 1 + 2A \cos \theta}{(A+1)^2} = \frac{1}{2} \left( 1 + \alpha + (1-\alpha) \cos \theta \right) \\
    \cos \psi &= \frac{1 + A \cos \theta}{\sqrt{A^2 + 1 + 2A \cos \theta}} \\
    \alpha &= \frac{(A-1)^2}{(A+1)^2} = \mbox{min ratio between final n energy and initial} 
  \end{align}

\item Lethargy, a unitless measurement of energy:
  \eqn{ u = \ln \frac{E_{\mathrm{ref}}}{E} }
Notice as time goes, neutrons slow down, $u$ increases, making it like a measure of the age of the neutrons. Then we can write:
\begin{align}
w &= u_f - u_i = -\ln \left( \frac{1}{2} ( 1 + \alpha + (1-\alpha) \cos \theta ) \right) \\
w_{min} &= 0 \\
w_{max} &= \epsilon = -\ln \alpha \\
P(w) \dw &= \frac{e^{-w}}{1 - \alpha} \dw \\
\expect{w} &= \xi = 1 - \frac{\alpha \epsilon}{1 - \alpha} 
\end{align}
$\xi$ is the average `progress' of the neutrons in terms of lethargy on the path of slowing down. That is, neutrons advance by $\xi$ lethargy units on average at each collision. Then to overcome the total lethargy interval $U = \ln \frac{E_0}{E_1}$, the number of collisions needed is:
\eqn{ n = \frac{U}{\xi} }

\item Moderating power is the best measure of a material's ability to slow down neutrons. It has two forms:
\eqn{\mbox{per atom basis} = \xi \sigma_s \fsp \fsp \fsp \mbox{per volume basis} = \xi \Sigma_s   }
A good moderating material should have: high slowing down (hence light nuclei), low capture (D, Be, C), moderating power (take into account both high slowing down and high scattering xs). 

\item Laws of inelastic collision. Elastic scattering is important for moderators because they are light; inelastic scattering is important for heavy materials like the fuel because they have almost no elastic collision. Minimum energy of the neutron for an inelastic collision is:
\eqn{ E_{\mathrm{threshold}} = \frac{A+1}{A} Q } 

\item First form of the slowing down equation:
  \begin{align}
    \rho(u) \du 
    &= \mbox{arrival density} = \begin{array}{l}
      \mbox{\# neutrons arriving per time and per volume} \\
      \mbox{in d$u$ between $u$ and $u + \du$ following a scattering to $u'$} 
      \end{array} \\
    &= \overbrace{\int_{-\infty}^u \Sigma_s (u') \Phi(u') \du'}^{\textcircled{1}} \overbrace{ P(u'\to u) \du }^{\textcircled{2}} \\
    \textcircled{1} &= \mbox{\# neutrons travelling in du' and scattered per time and per volume} \\
    \textcircled{2} &= \mbox{probability a neutron scattered at u' will be transferred in du} \\
    \rho(u) &= \int_{-\infty}^u \Sigma_s (u'\to u) \phi(u') \du' \\
    S(u) + \rho(u) &= \boxed{S(u) + \int_{-\infty}^u \Sigma_s (u'\to u) \phi(u') \du  = \Sigma(u) \phi(u) }
  \end{align}


\end{enumerate}





\lecture{Facts For Qualify Exam}

flux = $\frac{n}{\cm^2 \s}.$

Fast flux in hydrogen is around $10^{14}$ n/cm$^2$s, and on the order of $10^{12}$n/cm$^2$s for thermal flux. 

\end{document}
