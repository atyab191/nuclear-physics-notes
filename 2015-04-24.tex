\documentclass{school-22.211-notes}
\date{May 18, 2013}

\begin{document}
\maketitle

\lecture{Review: Ch 9 from Handbook of Nuclear Science \& Engineering}

\uline{\textbf{8 questions you should know from Ch 9}}:

\begin{enumerate}
\item LWR: roughly how many pins? how many axial zones a pin?

\bigskip 
  
  \item What are the typical steps in a modern lattice calculation to
  compute homogenized, two-group cross sections? 

\bigskip 

  
\item What are the main methods for resonance calculations? What are
  their pros and cons?

\bigskip 
  
  
\item How does NJOY work? What does xs in xs table depend on?
  Hint: infinitely dilute xs, background xs.

\bigskip 

  
\item How is resonance interference effect accounted for in lattice
  physics codes?

\bigskip 

  
\item What are some of the pros and cons of: CP, PN, SN, and MOC?

\bigskip 

  
\item How is leakage effect treated when we apply data generated by
  lattice physics code into nodal code?

 \bigskip 
 
\item What are the three most important burnable absorber materials in
  LWRs? Can you estimate how many nuclides we need to track in a 3D
  calculation?
\end{enumerate}


\clearpage
\textbf{Answers:} the page reference below are from Ch.9 of Handbook
of Nuclear Engineering, where page 1 is the start of Ch. 9 which is
labeled p.913 in the book. 

\begin{enumerate}
\item LWR: roughly how many pins? how many axial zones a pin?

  \textbf{Answers:} 15,000 to 20,000 pins. Typical bundle has seven or
  eight axial zones.

\item What are the typical steps in a modern lattice calculation to
  compute homogenized, two-group cross sections?

  \textbf{Answers:}Fig.3, also Fig.10 from p.58.

\item What are the main methods for resonance calculations? What are
  their pros and cons?

  \textbf{Answers:} Ultrafine energy groups, equivalence theory, and
  subgroup method (p.61). See table 7 (p.64) for pros \& cons of these
  methods. p.111 for more limitations of equivalence theory.

\item How does NJOY work? What does xs in xs table depend on?
  Hint: infinitely dilute xs, background xs.

  \textbf{Answers:} Eq. 3 on p.12:

  \eqn{ \sigma_{g,x,iso} =\sigma_{g,x,iso} (300K, \infty) \cdot
    f_{g,x,iso} (T, \sigma_0)}

  Also see section 1.4.1.

\item How is resonance interference effect accounted for in lattice
  physics codes?

  \textbf{Answers:} See section 1.4.2, we apply correction factors on
  top of NJOY-generated xs.

\item  What are some of the pros and cons of: CP, PN, SN, and MOC?

  \textbf{Answers:} see p.20.

  CP is good for small problems, because execution times and
  memory increases with square of \# meshes.

\item How is leakage effect treated when we apply data generated by
  lattice physics code into nodal code?

  \textbf{Answers:} Fundamental mode calculation / buckling
  calculation on p.23: find the $B^2$ such that $\keff = 1$ (do it
  iteratively).

\item What are the three most important burnable absorber materials in
  LWRs? Can you estimate how many nuclides we need to track in a 3D
  calculation?

  \textbf{Answers:} Boron, gadolinium, erbium (p.252). $10^9$ (p.252).
\end{enumerate}


\clearpage
\end{document}
