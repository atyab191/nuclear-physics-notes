\documentclass{school-22.211-notes}
\date{May 22, 2012}

\begin{document}
\maketitle


\lecture{Exam 3 Review: Dynamics \& Kinetics}
\begin{enumerate}
\item Fission happens very quickly but depletion effects are felt on much longer time scales. 
  \begin{itemize}
    \item Fission: $\sim 10^{-12}$s. 
    \item Prompt neutron life time: $\sim 10^{-5 \sim -4}$s. 
    \item Temperature effects: fuel $\sim 10^{-2}$s, coolant $\sim 100$s. 
    \item Fuel depletion: Xe $\sim$ hours, Pu$\sim$ days, fuel depletion is on the order of months.
    \item Spent fuel decay: $\sim$ 1000s years. 
  \end{itemize}

\item Delayed neutrons concepts: 
  \begin{itemize}
  \item Burst measurement: prompt neutrons; saturated: total neutrons. 
  \item Dominant FP that emits delayed neutrons: Br-37. 
  \item Delayed yields depend on: fissioning species, neutron energy.  
  \item Absolute yield: number of delayed neutrons per fission; unit: neutrons/fission. Eg: U238 = 4\%, U235=1.7\%. 
  \item Relative yield: absolute yield of an isotope divided by total absolute yield by all isotopes; unit: percentage. 
  \item Delayed neutron fraction: absolute yield divided by $\bar{\nu}$. Number of delayed neutrons devided by number of total fission neutrons. 
  \item Energy aspect of delayed neutron spectra: average prompt neutron emission energy: 2 MeV for prompt neutrons, 0.4 MeV for delayed neutrons. Both spectrum are Maxwellian, except delayed neutrons are more thermal, which makes them more likely to fission. 
  \item Energy group aspect: Delayed neutron spectra vary very slightly for different fissioning nuclides, but very significantly on energy groups. 
  \end{itemize}


\item Point kinetics equations: PKEs assume average properties, assume flux spatial shape is constant. 
  \begin{enumerate}
  \item Derivations: start with the time-dependent diffusion equation; assume flux is separable $\phi(\vecr, E, t) = S(\vecr, E) T(t)$ ($S$ is the shape function, $T$ is the amplitude function); integrate over all space and energy, normalize; define terms to simplify. Final form: 
    \eqn{ \Aboxed{ \ddt T(t) &= \frac{\rho(t) - \Sum_i \beta_i (t) }{\Lambda (t)} T(t) + \Sum_i \lambda_i C_i (t) + Q(t) } }
    \eqn{ \Aboxed{ \ddt C_i (t) &= \frac{\beta_i (t)}{\Lambda (t)} T(t) - \lambda_i C_i (t) }  }

    \item Physical terms: 
      \begin{enumerate}
      \item Reactivity $\rho$, where the top is the diffusion equation, and the bottom is fission rate if all the neutrons show up instatenously. 
        \scriptsize
        \begin{align}
          \rho (t) &= \frac{\int \bsp \dE \! \int \bsp \dr \Bigg\{ \!\divergence \! D  \gradient S  - \! \Sigma_t S  + \! \int \bsp \Sigma_s (\vecr \!, E' \bsp \to \bsp E, t) S \! \dE'   
            + \! \Sum_j \left( \! \chi_p^j (E) (1 \! - \! \beta^j) +  \! \Sum_i \chi_d^i (E) \beta_i^j \! \right) \! \int \bsp \nu \Sigma_f^j  S \dE'  }{\int \bsp \dE \int \bsp \dr \Sum_j \left( \chi_p^j (E) (1-\beta^j) + \Sum_i \chi_d^i (E) \beta_i^j \right) \int \bsp \nu \Sigma_f^j S \dE'} 
        \end{align}
        \normalsize
        \eqn{  \rho(t) &= \frac{\keff - 1.0}{\keff}  } 

      \item The delayed neutron fraction, 
        \begin{align}
          \beta_i (t) &= \frac{\int \bsp \dE \int \bsp \dr \left[ \Sum_j \chi_d^j (E) \beta^j \int_0^{\infty} \nu \Sigma_f^j (\vecr, E', t) S(\vecr, E') \dE' \right] }{\int \bsp \dE \int \bsp \dr \Sum_j \left( \chi_p^j (E) (1-\beta^j) + \Sum_i \chi_d^i (E) \beta_i^j \right) \int \bsp \nu \Sigma_f^j S \dE'} 
        \end{align}

      \item Prompt neutron lifetime, which is $\sim 1/v$ of shape divided by almost-instantenous-fission-rate. 

        \begin{align}
          \Lambda (t) &= \frac{\int \bsp \dE \int \bsp  \dr \left[ \frac{1}{v} S(\vecr, E) \right] }{\int \bsp \dE \int \bsp \dr \Sum_j \left( \chi_p^j (E) (1-\beta^j) + \Sum_i \chi_d^i (E) \beta_i^j \right) \int \bsp \nu \Sigma_f^j S \dE'} 
        \end{align}
        \eqn{ \Aboxed{ \Lambda(t) &\approx \frac{1}{v \nu \Sigma_f} \xrightarrow{\mathrm{criticality}} \frac{1}{v \Sigma_a} } }

      \item The $i$-th precursor, 
        \eqn{ C_i (t) = \int \dE \int \dr C_i (\vecr, t) }
      \item The total external source, 
        \eqn{ Q(t) = \int \dE \int \dr Q (\vecr, t) }
      \end{enumerate}


    \item Data
    \item Delayed neutron precursors
    \item Spectra
  \end{enumerate}

\item Point kinetics equations: 
  \begin{enumerate}
    \item Prompt-jump approximation: 
      \begin{align}
        \ddt T(t) &\approx 0 = \frac{\rho - \beta}{\Lambda} T(t) + \lambda C(t) \\
        T(0^+) &= \frac{\Lambda \lambda}{\beta - \rho} C(0^+) = \frac{\Lambda \lambda}{\beta - \rho} \left[ \frac{\beta}{\Lambda \lambda} T_0 \right] \\
        \Aboxed{ \frac{T(0^+)}{T_0} &= \frac{\beta}{\beta - \rho} }
      \end{align}
    \item In-hour equation: basically assume solutions take the form of, $T(t) \approx C_i (t) \approx e^{\omega t}$, plug into PKE, and get
      \eqn{ \Aboxed{ \rho &= \omega \Lambda + \Sum_i \frac{\beta_i \omega}{\omega + \lambda_i}} }

    \item Asymptotic periods: we can only use Inhour equation for asymptotic period, that is, a stable condition indicated by a constant period. Thus to use Inhour equation, we have to make the reactor stable first. For any time dependent problem, we have to use the more general IK. 
  \end{enumerate}

\item Reactivity units: 

\begin{table}[ht]
  \centering
  \begin{tabular}{|c|c|c|} \hline
    Unit & Definition & Example \\ \hline
    $\Delta k$ & actual units of PKEs & 0.01 \\
    \% $\Delta k$ & & 1\% \\
    pcm & $10^5 \Delta k $ & 1000 pcm \\ \hline
    Dollars & $\frac{\Delta k}{\beta}$ & \$1.5 \\
    Cents & 100 Dollars & 150 cents \\ 
    Milli-beta & 1000 Dollars & 1500 milli-beta \\ \hline
  \end{tabular}
\end{table}

\item Inverse kinetics equations: 1st order ODE again. First given $T(t)$ we solve for the precursor concentration,
\eqn{ \Aboxed{ [C(t)] &= [e^{-At}][C_0] + [e^{-At}] [A]^{-1} \left( [e^{At}][Y(t)] - [Y_0] \right) } }
where $[A] = \mathrm{diag}[\lambda_i], [Y(t)] = \frac{\mathrm{diag}[\beta_i]}{\Lambda} T(t)$. Then manipulating PKEs, we can get $\rho(t)$ in terms of $T(t), C(t)$, 
\begin{align}
\Aboxed{\rho(t) &= \frac{\Lambda}{T(t)} \ddt T(t) + \beta - \frac{\Lambda}{T(t)} \Sum_i \lambda_i C_i(t) - \frac{\Lambda}{T(t)} Q(t) } \\
\Aboxed{ \rho_n &= \frac{\Lambda}{T_n} \frac{T_n - T_{n-1}}{t_n - t_{n-1}} + \beta - \frac{\Lambda}{T_n} \Sum_i \lambda_i C_{i,n} - \frac{\Lambda}{T_n} Q_n } 
\end{align}

\item Fuchs-Nordheim model for RIAs: nice for intuition
  \begin{enumerate}
    \item Deposited energy: general expression, 
      \eqn{ P(t) &= P_0 + \frac{1}{\Lambda} \left[ C_p (\rho_{rod} - \beta) T_{fuel} (t)  - \frac{\alpha C_p}{2} T^2_{fuel} (t) \right]  }
      At peak power, we set $\frac{\dP}{\dT_{fuel}} = 0$, 
      \begin{align}
        \Aboxed{ T_{fuel}^{peak} &= \frac{\rho_{rod} - \beta}{\alpha} } \\
        \Aboxed{ P^{peak} &= P_0 + \frac{C_p (\rho_{rod} - \beta)^2}{2 \Lambda \alpha} }
      \end{align}

    \item Temperature behavior, Doppler coefficients. We can find the temperature when $P(t)$ returns to $P_0$ to be,     
      \eqn{ \Aboxed{ T_{fuel}^{end} &= \frac{2 (\rho_{rod} - \beta)}{\alpha} + T_{fuel}^0 } }

    \item Take-away messages:
      \begin{itemize}
      \item $\Delta T_{fuel}^{end} = 2 \Delta T_{fuel}^{peak}$. 
      \item F-N model is a good estimation because temperature is basically integrated power, hence the speed of reactivity insertion does not matter, it is the amount of reactivity inserted over time that matters. 
      \item Using PKEs with feedback, we demonstrate that reactivity change rate $\down$, the smaller the overshoot is. Though teh asymptotic results are independent of the reactivity change rate (as also seen in the F-N model). 
      \item Fuel temperature is independent of: reactivity insertion rate, neutron lifetime, heat capacity. 
      \end{itemize}
  \end{enumerate}

\item Transient feedback effects
  \begin{enumerate}
  \item Time constants: recall the asymptotic power decay time constant is just that of the longest precursor. 
  \item Power level changes: power overshoot is proportional to the reactivity change rate, but not the asymptotic value. 
  \end{enumerate}

%%%%%%%%%%%%%%%%%% Nodal %%%%%%%%%%%%%%%%%%%%
\item Nodal methods\footnote{weighted residual is not on the final}. The point of Nodal methods is that we can achieve high accuracy with large node size (compared with finite difference methods). We turn a 3D PDE into 3 1D ODE that are coupled through averaged transverse leakage terms. 
    \begin{align*}
      \boxed{ - \Sigma_{Dg}^x \frac{\derivative^2}{\dxi_x^2} \bar{\psi}_{gx} (\xi_x) + \Sigma_{rg} \bar{\psi}_{gx} (\xi_x) = \frac{1}{\keff} \chi_g \Sum_{g'=1}^G \nu \Sigma_{fg'} \bar{\psi}_{g'x} (\xi_x) + \Sum_{g'=1}^G \Sigma_{sg'g} \bar{\psi}_{g'x} (\xi_x) - L_{gy} (\xi_x) - L_{gz} (\xi_x) }    \\
   \boxed{ - \Sigma_{Dg}^y \frac{\derivative^2}{\dxi_y^2} \bar{\psi}_{gy} (\xi_y) + \Sigma_{rg} \bar{\psi}_{gy} (\xi_y) = \frac{1}{\keff} \chi_g \Sum_{g'=1}^G \nu \Sigma_{fg'} \bar{\psi}_{g'y} (\xi_y) + \Sum_{g'=1}^G \Sigma_{sg'g} \bar{\psi}_{g'y} (\xi_y) - L_{gz} (\xi_y) - L_{gx} (\xi_y) }  \\
     \boxed{ - \Sigma_{Dg}^z \frac{\derivative^2}{\dxi_z^2} \bar{\psi}_{gz} (\xi_z) + \Sigma_{rg} \bar{\psi}_{gz} (\xi_z) = \frac{1}{\keff} \chi_g \Sum_{g'=1}^G \nu \Sigma_{fg'} \bar{\psi}_{g'z} (\xi_z) + \Sum_{g'=1}^G \Sigma_{sg'g} \bar{\psi}_{g'z} (\xi_z) - L_{gx} (\xi_z) - L_{gy} (\xi_z) } 
    \end{align*}
  \begin{enumerate}
    \item NEM: expand $\bar{\psi}(\xi)$ by 4th order polynomial, 
          \eqn{ \bar{\psi}(\xi) = \Sum_{i=0}^4 a_i P_i (\xi) }

    \item Analytical: we solve 1D 2-group transerse-integrated diffusion equation, 
      \begin{align}
      -D_1 \frac{\derivative^2 \psi_1(x)}{\dx^2} + \Sigma_{r1} \psi_1(x) - \frac{1}{\keff} (\nu \Sigma_{f1} \psi_1 (x) + \nu \Sigma_{f2} \psi_2 (x) ) &= - L_1(x) \notag \\
      -D_2 \frac{\derivative^2 \psi_2(x)}{\dx^2} + \Sigma_{r2} \psi_2(x) - \Sigma_{12} \psi_1 (x) &= - L_2 (x)  \label{ANM-1}
    \end{align}
      We assume the solution is in the form of, 
      \eqn{ \psi_g (x) &= \hat{\psi}_g^H e^{iBx} + \psi_g^P(x) }
      such that $\dpsidxn2 = - B^2$. Plug in Eq.~\ref{ANM-1}, we get a system of two equations $[A] [\hat{\psi}^H] = 0$. For non-trival solution, we set det$[A] = 0$, thus getting a quadratic equation of $B^2$ that we can find the harmonic mode to be, 
      \eqn{ B_1^2&= b\left(-1 + \sqrt{1 - \frac{c}{b^2}} \right) = \left\{ \begin{array}{cc} > 0 & \kinf > \keff \\ < 0 & \kinf < \keff \end{array} \right. }
      and the harmonic mode to be, 
      \eqn{ B_2^2&=b\left(-1 - \sqrt{1 - \frac{c}{b^2}} \right) <0 }
      We need both modes because $B_1^2$ and $B_2^2$ are different. In this case fast group uses fundamental mode, thermal group uses harmonic mode, thus the solution is, 
      \eqn{ \left[ \begin{array}{c} \psi_1^H(x) \\ \psi_2^H (x) \end{array} \right] = \left[ \begin{array}{c} a_{11} \sin(B_1 x) + a_{12} \cos (B_1 x) \\ a_{21} \sinh(B_2 x) + a_{22} \cosh(B_2 x) \end{array} \right] =  \left[ \begin{array}{cc} r_1 & r_2 \\ 1 & 1 \end{array} \right] \left[ \begin{array}{c} a_{21} \sin(B_1 x) + a_{22} \cos (B_1 x) \\ a_{23} \sinh(B_2 x) + a_{24} \cosh(B_2 x) \end{array} \right] }
     where the fast-to-thermal flux ratio is defined as,
     \eqn{ r_m &= \frac{a_{11}}{a_{21}} = \frac{a_{12}}{a_{22}} = \frac{D_2 B_m^2 + \Sigma_{r2}}{\Sigma_{12}} }
     We know the particular solution from the $L$ terms. 

    \item SAMN: uses NEM for fast group, transverse integrated 1D diffusion equation for the thermal group. Approximate source with 4th order Legendre polynomial. Analytical solution of the thermal group is, 
      \eqn{ \psi_g(x) &= A \sinh(\kappa_g x) + B \cosh(\kappa_g x) + \Sum_{i=0}^4 c_i P_i \left(\frac{2x}{h}\right), &\kappa_g&=\sqrt{\frac{\Sigma_g}{D_g}} }
      That is, we have exponential homogeneous and polynomial particular solutions. 
  \end{enumerate}

\item Transverse-integration procedure
  \begin{enumerate}
    \item 3D to coupled 1D: we pick a direction of interest, perform integration within node over the plane normal to that direction, then devide by the planar area. For instance, the transverse leakage term is, 
          \eqn{ L_{gu}(\xi_x) &= \frac{1}{h_u} \left( \bar{J}_{gur}(\xi_x) - \bar{J}_{gul}(\xi_x) \right), & u &= y,z }
          where the line-averaged surface current is, 
          \eqn{ \bar{J}_{yr} (\xi_x) &= \int_0^1 J_y(\xi_x, 1, \xi_z) \dxi_z, &\bar{J}_{yl}(\xi_x) &= \int_0^1 J_y(\xi_x, 0, \xi_z) \dxi_z }
          \eqn{ \bar{J}_{zr} (\xi_x) &= \int_0^1 J_z(\xi_x, \xi_y, 1) \dxi_y, &\bar{J}_{zl}(\xi_x) &= \int_0^1 J_z(\xi_x, \xi_y,  0) \dxi_y }
          Flux distribution is not sensitive to transverse leakage shape, so we approximate transverse leakage with quadratic shape (2nd order polynomial): 
          \eqn{ L(\xi) = \bar{L} + l_1 P_1(\xi) + l_2 P_2(\xi) }
          and then use average TL conservation scheme to determine $l_1, l_2$. 
          
    \item Weighted residuals: in NEM, we have for each energy group 16 unknowns and 8 knowns, the rest 8 comes from weighted residual method: 
    \eqn{ \int_0^1 w(\xi) \left( -\Sigma_D \frac{\derivative}{\dxi^2} \psi(\xi) + \Sigma_r \psi(\xi) \right) \dxi = \int_0^1 w(\xi) \left( \frac{1}{\keff} \chi \psi(\xi) + S(\xi) - L(\xi) \right) \dxi }
    The closure relationship basically forces the $P_1, P_2$ integration to go to zero: 
    \eqn{ \int_0^1 P_1(\xi) \left( - \Sigma_D \frac{\derivative}{\dxi^2} \psi(\xi) + \Sigma_r \psi(\xi) - Q(\xi) \right) \dxi &= 0, & a_3 &= \frac{5 q_1 + 3 q_3 - 5 a_1 \Sigma_r}{3(60 \Sigma_D + \Sigma_r)} }
    \eqn{ \int_0^1 P_2(\xi) \left( - \Sigma_D \frac{\derivative}{\dxi^2} \psi(\xi) + \Sigma_r \psi(\xi) - Q(\xi) \right) \dxi &= 0, & a_4 &= \frac{-7 q_2 + 3 q_4 + 7 a_2 \Sigma_r}{420(\Sigma_D + 3\Sigma_r)} } 
  \end{enumerate}


\clearpage
%%%%%%%%%%%%%%%%%%% Homogenenization methods %%%%%%%%%%%%%%%%%%%%%%%%%%
\item Homogenization methods
  \begin{enumerate}
    \item \textbf{Discontinuity factors}: since the homogenized flux distribution in each node is affected by $D$, and the choice of the flux weighted $D$ is somewhat arbitrary, the interface fluxes can be different. We define DF: 
      \eqn{ f_{gi,j}^{u-} &= \frac{\phi_{gi,j}^u (u_l)}{\hat{\phi}_{gi,j}^u (u_l)}, &f_{gi,j}^{u+} &= \frac{\phi_{gi,j}^u (u_{l+1})}{\hat{\phi}_{gi,j}^u (u_{l+1})} }
      ADFs are the homogeneous analogy to the heterogeneous assembly calculation (which is calculating a single-node problem with zero net current BC). ADFs are simply ratios of the surface-averaged fluxes to the cell-averaged fluxes in the heterogeneous assembly calculation.

    \item Reflector modeling: 
      \begin{itemize}
      \item Using empiracal albedos to replace the baffle and reflector is bad because using flux-volume weighted cross section distribute the strong absorption xs over the entire region. 
      \item We use DFs because they are less spatially sensitive than albedos. It is advantagous in that DFs are chosen so that the nodal model wouold reproduce the net currents at the core/baffle interface and the net reaction rates in both the assembly and the homogenized baffle without explicitly representing the baffle. 
      \item Fast neutron leakage makes up for 90\% of all leakage, so leakage is insensitive to enrichment, burnup etc. 
      \end{itemize}
  \end{enumerate}

\item Pin power reconstruction
  \begin{enumerate}
    \item Form functions: we multiple homogeneous solution by the form function to get the heterogeneous solution. 
      \eqn{ \phi_g^{het} (x,y) &= \phi_g^{hom}(x,y) \phi_g^{SA} (x,y) }

    \item Corner point discontinuity factors. 
      \begin{itemize}
      \item Corner point fluxes are approximated by assuming that intra-nodal flux distributions are separable: 
        \eqn{ \phi_{g,i,j}(x,y) = \frac{\phi_{g,i,j} (x) \phi_{g,i,j} (y)}{\bar{\phi}_{g,i,j}} }
      \item Heterogeneous corner point flux has to be continuous. Thus we need corner point ratios that are analogous to DFs that can be obtained from the lattice calculations. 
      \item In general, the corner-point fluxes are determined by averaging the four estimates of the heterogeneous corner-point flux, 
        \eqn{ \phi_g^{het,cp} &= \frac{1}{4} \left[ \phi_{g,i,j}^{hom,cp} \phi_{g,i,j}^{fct,cp} + \phi_{g,i+1,j}^{hom,cp} \phi_{g,i+1,j}^{fct,cp} + \phi_{g,i,j+1}^{hom,cp} \phi_{g,i,j+1}^{fct,cp} + \phi_{g,i+1,j+1}^{hom,cp} \phi_{g,i+1,j+1}^{fct,cp} \right] }
      \item In order to satisfy the continuity conditions of reconstructed fluxes, the best-estimate homogeneous corner-point fluxes for each node are computed by, 
        \eqn{ \hat{\phi}_{g,i,j}^{hom,cp} = \frac{\phi_g^{het,cp}}{\phi_{g,i,j}^{fct,cp}} }
      \item  The reconstructed corner point fluxes are then continuous. \textbf{CP ratio is the corner point analogue of ADF for assembly edges.}
      \end{itemize}
  \end{enumerate}
  Side note: there are two entirely different problems. One is an eigenvalue problem, using homogenized cross section, and output $k$, and the flux computed would be continuous. The other is a fixed-source problem, using $\keff$ as input (hence $J$ at interface), and generate discontinuous flux. 


\clearpage
%%%%%%%%%%%%%%%%%%% Perturbation Theory %%%%%%%%%%%%%%%%%%%%%%%%%%
\item  First-Order Perturbation (FOP): the point is to solve for reactivity resulted from a perturbation without have to solving for the perturbed spatial flux. There are three sets of expressions, one for unperturbed forward problem, one for the unperturbed adjoint problem,
\eqn{ A_0 \psi_0 &= \frac{1}{k_0} M_0 \psi_0 &A_0^*\psi_0^* &= \frac{1}{k_0^*} M^* \psi_0^*  }
and a third set of perturbed forward problem, 
\eqn{ A \psi &= \frac{1}{k} M \psi \label{pertb3}}
FOP approximate the perturbed problem using, 
\eqn{ A &= A_0 + \delta A   &M &=M_0 + \delta M  &\psi &=\psi_0 + \delta \psi &k&=k_0+\delta k}
Plug the above approximation into Eq.~\ref{pertb3} and simplify, we reach the FOP which is 2nd order accurate, 
   \eqn{ \boxed{\delta \rho \approx \frac{\expect{\psi_0^*, \left( \frac{1}{k_0} \delta M - \delta A \right) \psi_0 }}{\expect{\psi_0^*, M_0 \psi_0}}  } }

\item Adjoint fluxes: if we put a neutron at this place this energy, what is the probability it would cause fission.
  \begin{enumerate}
  \item Balance equations:
    \begin{align}
      -\gradient D_g  \gradient \phi_g + \Sigma_{tg} \phi_g &= \chi_g \Sum_{g'=1}^G \nu \Sigma_{fg'}  \phi_{g'} + \Sum_{g'=1}^G \Sigma_{sg'\to g} \phi_{g'} \\
      -\gradient D_g  \gradient \phi_g^* + \Sigma_{tg}  \phi_g^* &= \nu \Sigma_{fg}  \Sum_{g'=1}^G  \chi_{g'} \phi_{g'}^*  + \Sum_{g'=1}^G \Sigma_{sg\to g'}  \phi_{g'}^*
    \end{align}

  \item For two group, infinite medium case, with effective downscatter only, 
    \begin{align}
      \left[ \begin{array}{cc}
          \Sigma_{a1} + \Sigma_{12} - \frac{1}{\kinf} \nu \Sigma_{f1} & - \frac{1}{\kinf} \nu \Sigma_{f2} \\
          - \Sigma_{12} & \Sigma_{a2} \end{array} \right] 
      \left[ \begin{array}{c}
          \phi_1 \\ \phi_2 \end{array} \right] &= 0  
      &\frac{\phi_2}{\phi_1} &= \frac{\Sigma_{12}}{\Sigma_{a2}}   \\
      \left[ \begin{array}{cc}
          \Sigma_{a1} + \Sigma_{12} - \frac{1}{\kinf^*} \nu \Sigma_{f1} & - \Sigma_{12} \\
          - \frac{1}{\kinf^*} \nu \Sigma_{f2} & \Sigma_{a2} \end{array} \right] 
      \left[ \begin{array}{c}
          \phi_1^* \\ \phi_2^* \end{array} \right] &= 0  
      &\frac{\phi_2^*}{\phi_1^*} &= \frac{\frac{1}{\kinf^*} \nu \Sigma_{f2}}{\Sigma_{a2}}  
    \end{align}
    where 
    \eqn{ \kinf^* = \kinf = \frac{\nu \Sigma_{f1} + \nu \Sigma_{f2} \frac{\Sigma_{12}}{\Sigma_{a2}} }{\Sigma_{a1} + \Sigma_{12}} } 
    In the above example, we effectively have $M \phi = 0, M^* \phi^* = 0$, and by definition, the adjoint of Hermitean conjugate of a mtatrix is just complex-conjugating each of its element and then transposing, 
    \eqn{ M^* = \left[ \begin{array}{cc} a_{11} & a_{12} \\ a_{21} & a_{22} \end{array} \right]^* = \left[ \begin{array}{cc} a_{11}^* & a_{12}^* \\ a_{21}^* & a_{22}^* \end{array} \right]^T = \left[ \begin{array}{cc} a_{11}^* & a_{21}^* \\ a_{12}^* & a_{22}^* \end{array} \right] }
 
  \item Boundary conditions: $\psi, \psi^*$ are any two functions satisfying the appropriate boundary and continuity conditions for the angular flux and adjoint flux. If $\psi(\vecr, \Omega, E) = 0$ for all $\hat{n} \cdot \Omegahat < 0$, then the adjoint satisfying $\psi^*(\vecr, \Omegahat, E) = 0$ for $\hat{n} \cdot \Omegahat > 0$. 


  \item Meaning:
    \begin{itemize}
    \item Fundamental mode solution: adjoint eigenvalue is the same as the real eigenvalue: $k^* = k$. 
    \item Adjoint flux is defined as the asymptotic increase in total neutron population of a critical reactor for a neutron introduced a phase space (position $r$, direction $\omega$, and energy $E$): 
      \eqn{\frac{N(\infty) - N_0}{Q} &= \psi^* (\vecr, \Omegahat, E)  }
    \end{itemize}

  \item Spectra: 
    \begin{itemize}
    \item The PWR spectrum: . 
    \item For a typical two group problem, the adjoint flux should have the same shape for the two groups, except the magnitude is offset by $\frac{\phi_2^*}{\phi_1^*}$. 
    \end{itemize}
  \end{enumerate} 

\item Adjoint fluxes in kinetics parameters\footnote{In operating reactors, they measure the control rod worth; they are no better than the delayed spectra}
  \begin{enumerate}
  \item Delayed spectra: 10 times that of the prompt for up to 0.1 MeV; then approahces prompt until smaller than prompt spectrum for 1 MeV and above. 

  \item $\bar{I}$ is the ratio of the adjoint weighted delayed chi to prompt chi, and it has a value of about 0.97.
   \eqn{ \bar{I} = \frac{\expect{S_0^* (E), \chi_d (E)}}{\expect{S_0^*(E), \chi_p (E)}} }
  \end{enumerate}

\item Loading pattern optimization
\end{enumerate}
%%%%%%%%%%%%%%%%%%%%%%%%% Final Exam End %%%%%%%%%%%%%%%%%%%%%%%%%%%%

\end{document}
