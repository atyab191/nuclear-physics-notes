\documentclass{school-22.101-notes}
\date{September 12, 2011}

\begin{document}
\maketitle

\topic{Four Quantum Mechanics Postulates}
A high-level overview of the four postulates, 
\begin{enumerate}
\item State. 
\item Physical properties/observables. 
\item Measurements. 
\end{enumerate}

\subtopic{Postulate 1: Wave Functions Live in Hilbert Space}
\begin{axiom} 
    The state function/state is a wave function that contains all knowledge pertained to that particle. 
\end{axiom}
Two other ways to state this postulate are: 
\begin{itemize}
\item The state of a physical system is described by a state vector $\ket{\psi}$. $\left\{ \ket{\psi} \right\}$ form a Hilbert space.
\item The properties of a quantum system are completely defined by specification of its state vector $\ket{\psi}$. The state vector is an element of a complex Hilbert space $H$ called the space of states.
\end{itemize}
A quantum state $\psi(\uline{x}, t)$ is complex in space and time, is normalized, and contains all knowledge (aka any physical behavior associated with the state) we may be interested in learning about the physical behavior. 


\clearpage
\subtopic{Postulate 2: Observable Are Represented By Hermitian Operators}
\begin{axiom} 
    The \uline{permissible} values of the observables are the eigenvalues from solving the associated mathematical operators.
\end{axiom}
Two other ways to state this postulate are:
\begin{itemize}
\item Physical properties (observables) are described by operators A. Eg., $\ket{\phi} = A \ket{\psi}$. 
\item With every physical property $A$ there exists an associated linear, Hermitian operator $\hat{A}$ called observable which acts in the Hilbert space $H$. The eigenvalues of the operator are the possible values of the physical properties.
\end{itemize}

\hi{Hermitian operators} mean that eigenvalues are real, and eigenvectors form an orthorgonal basis. Hermitian operators are always self-adjoint, because self-adjoint means $A = A^*$ which is always going to be true for operators with real eigenvalues. 


For example, a 1D free particle system can be solved using its operators to find the permissible values of the observables. 
\begin{enumerate}
\item \uline{Operators Definitions in 1D Free Particle Problem.} Position operators by definition are: $\hat{x}, \hat{y}, \hat{z}$ (or the vector $\hat{\underline{x}}$). Momentum operators by definition are:
  \eqn{ \hat{p}_x &= - i \hbar \ppx,  &\hat{p}_y &= - i \hbar \ppy, &\hat{p}_z &= - i \hbar \ppz, & \Aboxed{\hat{p} &=  -i \hbar \gradient} }

  Classically, energy is: $E = KE + PE = \frac{1}{2m} (p_x^2 + p_y^2 + p_z^2) + V(x,y,z)$. Quantum mechanically, the energy operators are:
  \eqn{ \hat{H} = \frac{1}{2m} \left( \hat{p}_x^2 + \hat{p}_y^2 + \hat{p}_z^2 \right) + V(\hat{x}, \hat{y}, \hat{z}) = \boxed{ - \frac{\hbar^2}{2m} \laplacian + V(\hat{x}, \hat{y}, \hat{z})} }



\item \uline{Position eigenvalue problem for 1D free particle.} The position eigenvalue problem is:
  \eqn{ \hat{x} \ket{u_n} &= x_n \ket{u_m}  }
  There are two important aspects of this (and any) eigenvalue problem: 
  \begin{enumerate}
  \item Describing the coefficients $C_n$. Using the bracekt notation for inner product, we can describe the state as (and keep in mind that the two different notations for continuous basis vs. discretized basis)\footnote{To keep it general, we use $C_n$ instead of $x_n$ for the eigenvalues for now.}. 
    \eqn{ \ket{\psi} = \int \dx C(x) \ket{u_n} = \Sum C_n \ket{u_n}  \label{ket-psi}}
    where 
    \eqn{ C_n = \braket{u_n}{\psi} \label{Cn}}


  \item Describing basis. In the above equation, $\left\{ \ket{u_n} \right\}$ are the basis for position operator. 
    Plug Eq.~\ref{Cn} into Eq.~\ref{ket-psi}, 
    \begin{align}
      \ket{\psi} &= \int \dx C(x) \ket{x} = \int \braket{x}{\psi} \ket{x} \dx \xrightarrow{\psi(x) = \ket{\psi}} \int \bra{x} \psi(x) \ket{x} \dx \\
      &\int x u_n(x)\ket{x} \dx = \int x_n u_n (x) \ket{x} \dx = \int x_n \delta (x-x_n) \ket{x} \dx 
    \end{align}
    \hi{Diract delta function $\delta(x-x_n)$ is the eigenfunction of position operator.} 

    \eqn{ \ket{u_n} &= \delta (x - x_n) }
    That is, $u_n(x_n) = \infty, u_n (x\neq x_n) = 0,$ and we can prove that the area under the delta curve to be 1: 
    \begin{align}
      \int_{-\infty}^{\infty} \dx f(x) \delta(x-x_n) &=  \int_{-\infty}^{\infty} \dx x \delta(x-x_n) = x_n = f(x) \\
      \Rightarrow \int_{-\infty}^{\infty} \dx \delta(x-x_n) &= 1 
    \end{align}
  \end{enumerate}

\item \uline{Momentum eigenvalue problem for 1D free particle.} The momentum eigenvalue problem is, 
\begin{align}
  \hat{P} \ket{u_n} &= p_n \ket{u_n}  &   \hat{P} u_n(\uline{x}) &= p_n u_n(\uline{x}) \\
  -i\hbar \ppx u_n(\uline{x}) &= p_n u_n(\uline{x}) &  \ppx u_n &= i \frac{p_n}{\hbar} u_n   \\
  u_n &= A e^{i \frac{p_n}{\hbar} x} = A e^{ikx} = A (\cos kx + i \sin kx) 
\end{align}
where we define $\boxed{k = \frac{p}{\hbar}}$ \footnote{$\hbar$ being the reduced Planck constant $\hbar = \frac{h}{2\pi}$. $h, \hbar$ are invented to convert between energy and wavelength after people realize that they are really the same thing with different units.}, and combine with the De Broglie wavelength $\lambda = \frac{h}{p}$, we can obtain the relation of $ \boxed{k \lambda = 2 \pi}$ two ways, 
\eqn{k &= \frac{p}{\hbar} = \frac{2\pi}{\lambda} & \lambda &= \frac{h}{p} = \frac{2 \pi \hbar}{p} =  \frac{2\pi}{k}}
\textcolor{blue}{The eigenfunction of the momentum eigenvalue problem gives us a de Broglie wave traveling to the right hand, and the momentum value, or the eigenvalue of the problem is $p_n = \hbar k$.} Note this is a continuous solution, meaning there is no restriction on wavelength $\lambda$ or wave number k, because this is a 1D free particle problem. 

\textbf{Example:} Is $\cos kx$ a wave function satisfying the momentum eigenvalue problem? \\
\textbf{Answer:} We can consider:
\eqn{ \cos kx = \frac{1}{2} (e^{ikx} + e^{-ikx})  }
This is to say the eigenfunction $\cos kx$ is actually the combination of a wave traveling to the right $e^{ikx}$ and a wave traveling to the left $e^{-ikx}$, and it is not the solution of the momentum eigenvalue problem, hence it does NOT satisfy the momentum eigenvalue problem. 


\item \uline{Energy eigenvalue problem for 1D free particle.} The energy operator is Hamiltonian: 
\eqn{ \hat{H}  = \frac{\hat{p} \cdot \hat{p}}{2m} + \hat{V}  = -\frac{\hbar^2}{2m} \gradient^2 + V(x,y,z)}
For free particles, we can drop the potential term. The energy eigenvalue problem becomes,
\begin{align}
  \hat{H} \ket{u_n} &= E_n \ket{u_n} &  \hat{H} u_n (\uline{x}) &= E_n u_n (\uline{x} ) \\
  -\frac{\hbar^2}{2m} \gradient^2 u_n (\uline{x}) &= E_n u_n (\uline{x} )  &  \ppxn2 u_n + \frac{2m E_n}{\hbar^2} u_n &= 0 \\
 \ppxn2 u_n + k^2 u_n &= 0   & u_n &= A e^{ikx} + B e^{-ikx} 
\end{align}
The solution can be written in two ways, 
\begin{align}
  u_n = \left\{ 
  \begin{array}{cc}
    A e^{ikx} + B e^{-ikx} & \mbox{Travelling Wave} \\
    A' \sin kx + B' \cos kx  & \mbox{Stationary Wave} 
  \end{array}
  \right.
\end{align}
Keep in mind that we replaced $\frac{2m E_n}{\hbar^2}$ with $k_n^2$ because:
\eqn{ p = \hbar k \Rightarrow k = \frac{p}{\hbar} = \frac{\sqrt{2 m E}}{\hbar} }
Keep in mind that for free particles there is no restriction on the $k$, position or anything else. 


\item \uline{Summary}
\begin{table}[ht]
\begin{tabular}{|l|p{1.6in}|l|p{2in}|} \hline
Observables & Operators & Eigenvalues & Eigenfunctions \\ \hline
Position & $\hat{x} = x$ & $x_j$ (no constrain) & $\delta(x-x_j)$ \\ \hline
Momentum & $\hat{p} = - i \hbar \gradient $ & $p_n = \hbar k$ (no constrain) & $u_n = A e^{i \frac{p}{\hbar} x } = A e^{i k x} $ \\ \hline
Energy & $\hat{H} = \frac{\hat{p}^2}{2m} + V = - \frac{\hbar^2}{2m} \laplacian + V$ & $E_n = \frac{\hbar^2 k^2}{2m}$ (no constrain) & $W_n = A \sin kx + B \cos kx = A^{\prime} e^{ikx} + B^{\prime} e^{-ikx} $\\ \hline
\end{tabular}
\caption{Properties of Position, Momentum, Energy Operators}
\end{table}

\hi{If a function is the eigenfunction of the momentum operator, it must also be the eigenfunction of the energy operator.} Proof:
\begin{align}
\hat{p} W_n &= \hbar k W_n \\
\hat{H} W_n &= \frac{\hat{p}^2}{2m} W_n 
= \frac{\hat{p}}{2m} (\hat{p} W_n) 
= \frac{\hat{p}}{2m} \hbar k W_n = \frac{\hbar k}{2m} \hat{p} W_n = \frac{\hbar^2 k^2}{2m} W_n = E_n W_n
\end{align}
Thus for any $W_n$ that is the eigenfunction of $\hat{p}$, it must also be the eigenfunction of $\hat{H}$. 
\end{enumerate}
To summarize the second postulate, basicaly we relate physical observables (position, momentum, energy) with their mathematical operator, through which we solve for the possible values for the observables.
\end{document}
