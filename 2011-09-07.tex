\documentclass{school-22.101-notes}
\date{September 7, 2011}

\begin{document}
\maketitle


\lecture{Quantum Mechanics Fundamendals}
Nucleus (protons \& neutrons) are the smallest units in this course. 

\uline{Two problems in this course:}
\begin{enumerate}
\item (statics) Properties of force that holds neutrons and protons together. 

Sample sections: nuclear structure, stability (scattering). 

\item (dynamics) Behavior of atoms of many particles. 

Sample sections: nuclear decay, neutron interactions, nuclear reaction. 
\end{enumerate}

\uline{Nuclear Nomenclature}
\begin{enumerate}
\item Protons and neutrons are around the size of 1 femmi ($10^{-13}$ cm, or $10^{-15}$ m). 

\item Mass is important because it has to do with energy and stability. For now (without regarding to stability), we can consider: 
\eqn{ A = N + Z = \mbox{Mass number}}
because $m_e \ll m_p, m_n$. 

\item Nuclear size $\approx 10^{-13}$ to $10^{-12} \cm \approx 10^{-5}$ to $10^{-4} \AA$. 

Atomic size $\approx 10^{-8} \cm \approx 1 \AA$. 

Example: Fe $\approx 2-3 \AA$. 

\item The force that combines protons and neutrons is: strong nuclear force. Protons and neutrons are held to their first neighbors tightly together to form a roughly spherical shape. 

\item Discovery of nucleus: 

Prior to the discovery of nucleus around 1900, the guess is the `plumb pudding model.' Two groups of people shot alpha-particles into solid matters (Geiger \& Madsen in 1909, and Rutherford in 1911). They found that most particles went through, and only very small fraction (1/8000) got reflected back ($\theta > 90^{\degree}$). 

The difference between Thomson's model and Rutherford's is that Rutherford's has positive charges densely packed in the middle, thus the small fraction of alpha particles that interact with the protons experience large Coulomb repulsion force, resulting in large scattering angle. 

Example relating KE to Coulomb potential energy V: run experiment and found that the alpha particles with the largest KE targeting Uranium is about 20 MeV (equivalent to $3.204 \times 10^{-12} \J$). This is the largest KE of alpha particles getting close to uranium atoms without penetrating through them. Thus we plug in to the following: 
\eqn{ V_{\alpha} = k_e \frac{ (2e) (92e)}{D} \Rightarrow D = \frac{k_e 184 e^2}{V_{\alpha}} = \frac{9 \times 10^9 \times 184 \times (1.6 \times 10^{-19} )^2}{3.204 \times 10^{-12} \J} = 13 F \approx R_{Uranium} + R_{\alpha} = 9 F + 2 F = 11 F }
Notice the $V_\alpha$ is the potential energy, which is in unit of energy. 


\item $r_n \approx r_e \approx r = 1.2 F, Z + Z = A. V = A \frac{4\pi}{3} r^3 = \frac{4 \pi}{3} R^2 \Rightarrow R^3 = A^{1/3} r.$ Hence,
\eqn{  \boxed{R (\mbox{in F}) = (1.2 \sim 1.4) A^{1/3}}}
The 1.2 is from electron scattering with protons, which depends on distribution of charges. The 1.4 is from neutron scattering which depends on the extent of nuclear force, and depends on the neutrons and protons in the core, as well as the neutron skin. 

The above expression suggests that the size of a nucleus is related to the atom number. 

The density of nucleus is about $10^{14} \g/\cm^3$, which corresponds to about $10^{38} $ nucleons/$\cm^3$. 


\end{enumerate}


\topic{Nuclear Stability}
To start with, we need to clarify a few terms:
\begin{enumerate}
\item Nucleon: a collective name for two particles: the neutron and the proton. 
\item Nucleus (plural nuclei): the very dense region consisting of protons and neutrons at the center of an atom.
\item Nuclide: an atomic species characterized of its nucleus, like number of protons (called isotopes), number of neutrons (isotones), mass number (isobars) etc. 
\end{enumerate}

There are a few important points:
\begin{enumerate}
\item \# unstable nuclides $\gg$ \# stable nuclides.
\item Z vs N curve: For larger atoms to be stable, they need to have more protons than neutrons. 
\item To tell whether a nucleus is stable or not: $\Delta = M_{\mathrm{nucleus}} - \Sum_A m_{n,p} $. The range of A (or Z) that results in a negative $\Delta$ is the range of stability. The reason $\Delta$ is concave up will be discussed in the future.
\item Classically, we know $V_{\mathrm{coul}} = \frac{q_1 q_2}{r} = \frac{Z_1 Z_2 e^2}{r}$. That is, classically there is a minimum energy E that alpha particles need to penetrate through. But in an alpha decay \ce{ U238 \to \alpha + Th234} the experimental $E_{\alpha} = 4.2$ MeV does not agree with the predicted $V_{max} = 8.6$ MeV (which is the potential of the barrier).

The probability of alpha decay $P^{\alpha \to} \neq 0$. The explanation is quantum tunneling. We need a set of principles that explain the physical behavior of the very small particles, like atomic (atoms, molecules), subatomic (electron, neutron, proton, etc). 

While we are measuring any observable, they will be associated with a probability -- so we are not going to measure/predict an observable with certain precision, but rather with probability. 

\end{enumerate}




\end{document}
