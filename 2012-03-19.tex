\documentclass{school-22.211-notes}
\date{March 19, 2012}

\begin{document}
\maketitle

\lecture{Multi-Group Diffusion: Derivations}
This class would not require derivation of the transport theory. 
\topic{Derivation Of Diffusion Theory From Transport Theory}
\begin{enumerate}
\item Start from the steady state continuous energy transport equation: 

\item Expand scattering source/flux in spherical harmonics, integrate over angle,


The scattering kernel is nasty, because a) $\Sigma_s$ depends on energy, hence we have to integrate over all energy; b) $\Sigma_s$ depends on angle, hence we have to integrate over all angle; c) $\Sigma_s$ depends on isotope, hence we have to integrate over all isotopes. 

\item Simplify terms: 


\item We obtain the balance equation:

Source equals sink. Notice this equation is exact so far. But since we do not know what $\vec{J}$ is, we need to approximate it somehow. 

\item Start from transport equation again, multiple by $\Omega$ and integrate over all angles: 


\item Simplify terms:

\item We obtain the net current equation:

\item We assume the source is isotropic, then the source term goes away:

\item We assume the scattering is isotropic in the Lab system, then the scattering kernel goes away:

Notice this assumption is not very good in a typical PWR -- because hydrogen scattering dominates, and hydrogen scatters isotropically in the COM system, not in the lab system. 

\item Correction term: \hi{transport correction $P_0$} approximation: 

\end{enumerate}


\clearpage
\topic{Multi-Group Diffusion Theory}

The diffusion coefficient approximation table is very important: $\frac{1}{3\Sigma_t}$ is off by a factor of 3! 



The group diffusion coefficient $D_g$ should be a tensor; but in application, we assume that $D_g$ is the same for all directions. A more accurate approach is to let $D_g$ be the same in the x-y plane, but different in axial direction. 




\clearpage
\topic{Applications of Diffusion Theory}
\begin{enumerate}
\item 3D Noval Approach: solve unique lattice in detail; treat each assembly as homogenized; solve 3D homogenized-asssembly few-group diffusion problem. 
\item 3D Pin Approach: solve unique pins or lattice in detail; treat each pin as homogenized; solve 3D homogenized pin few-group diffusion problem. 
\item [FIXME]
\end{enumerate}

\topic{Continuity Conditions}


\topic{Partial Currents}


\hi{Albedo boundary condition} is the measurement of how much flux is reflected back: 
\eqn{ \alpha = \frac{J^-}{J^+} }


\topic{Boundary Conditions For Diffusion Theory}
In diffusion theory, the incoming partial flux is zero:
\eqn{ J^- = \frac{1}{4} \phi - \frac{1}{2} J_n = 0 }
There are two formulisms to get the boundary condition: 
\begin{enumerate}
\item 
\eqn{ \frac{J}{\phi} = \frac{1}{2} }
\item The more conventional boundary condition is: 
\begin{align}
  J^- &= \frac{1}{4} \phi + \frac{D}{2} \gradient \phi_n \\
  \frac{\gradient \phi}{\phi} &= - \frac{1}{2D} = - \frac{3\Sigma_{tr}}{2} = - \frac{1}{d_{\mathrm{extrap}}}, \mbox{ where } d_{\mathrm{extrap}} = \frac{2}{3} \Sigma_{tr}  = \frac{2}{3 \lambda_{\mathrm{tr}}} 
\end{align}
where $D$ is the property of the material inside, and $\lambda_{tr}$ is transport mean free path. Notice that the exact coefficient in $d_{\mathrm{extrap}}$ may be different. 
\end{enumerate}


\topic{Eigenvalue Problem}
We start from a one-group diffusion equation,
\eqn{ - \divergence D(\vecr) \gradient \phi(\vecr) + \Sigma(\vecr) \phi(\vecr) = \frac{1}{\keff} \nu \Sigma_f(\vecr) \phi(\vecr) }
Assume cross sections do not depend on spatial distribution, 

Rearranging, 

Define \hi{material buckling} 
\eqn{ B^2_m = }
Then we get our \hi{Helmholtz equation}:


For a criticality problem, we can solve find the $B^2$ such that the system is critical. Hence giving geometry, only certain materials would make the system critical; given material, only certain geometries would make the system critical. That is, given a reactor, it would be critical if and only if the material and the geometry satisfy,
\eqn{ B_g^2 = B_m^2} 
and there can only be a solution when $\kinf > 1$ (exception: if there is external source, $\kinf$ can be less than 1 and the system is still critical). 


$DB^2_g$ is the leakage per unit volume per unit flux. Notice $\keff$ does not depend on volume or flux. 


\hi{migration area} $M^2 = \frac{D}{\Sigma_a}$ is a measurement of the amount of travelling before absorption. 


\end{document}
