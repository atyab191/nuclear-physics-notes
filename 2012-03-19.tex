\documentclass{school-22.211-notes}
\date{March 19, 2012}

\begin{document}
\maketitle

\lecture{Diffusion Theory}
This class would not require derivation of the transport theory. 
\topic{Derivation Of Diffusion Theory From Transport Theory}
\begin{enumerate}
\item Start from the steady state continuous energy transport equation: 

\item Expand scattering source/flux in spherical harmonics, integrate over angle,


The scattering kernel is nasty, because a) $\Sigma_s$ depends on energy, hence we have to integrate over all energy; b) $\Sigma_s$ depends on angle, hence we have to integrate over all angle; c) $\Sigma_s$ depends on isotope, hence we have to integrate over all isotopes. 

\item Simplify terms: 


\item We obtain the balance equation:

Source equals sink. Notice this equation is exact so far. But since we do not know what $\vec{J}$ is, we need to approximate it somehow. 

\item Start from transport equation again, multiple by $\Omega$ and integrate over all angles: 


\item Simplify terms:

\item We obtain the net current equation:

\item We assume the source is isotropic, then the source term goes away:

\item We assume the scattering is isotropic in the Lab system, then the scattering kernel goes away:

Notice this assumption is not very good in a typical PWR -- because hydrogen scattering dominates, and hydrogen scatters isotropically in the COM system, not in the lab system. 

\item Correction term: \hi{transport correction $P_0$} approximation: 

\end{enumerate}




\end{document}
