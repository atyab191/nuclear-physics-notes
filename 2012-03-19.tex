\documentclass{school-22.211-notes}
\date{March 19, 2012}

\begin{document}
\maketitle

\lecture{One-Group Diffusion: Analytical Solutions}
\topic{Helmholtz Equation}
We start with the one-group diffusion equation,
\eqn{ -\divergence D(\vecr) \gradient \phi(\vecr) + \Sigma_a(\vecr) \phi(\vecr) = \frac{1}{\keff} \mu \Sigma_f(\vecr) \phi(\vecr) }
Assume homogeneous material, that is, spatial constant cross section,
\eqn{ -D \laplacian \phi(\vecr) + \Sigma_a \phi(\vecr) = \frac{1}{\keff} \nu \Sigma_f \phi(\vecr) }
Rearranging and defining a buckling term, 
\eqn{ \laplacian \phi(\vecr) + \underline{\frac{ \frac{\nu \Sigma_f}{\keff} - \Sigma_a}{D} }_{B^2} \phi(\vecr) = 0 }
We obtain a classic Helmholtz Equation,
\eqn{ \boxed{\laplacian \phi(\vecr) + B_m^2 \phi(\vecr) = 0 } }
Helmholtz Equation implies that,
\eqn{ B_m^2 = - \frac{\laplacian \phi(\vecr)}{\phi(\vecr)} }
Notice,
\begin{enumerate}
\item Since $B_m^2$ is a constant, that is to say $\frac{\laplacian \phi(vecr)}{\phi(\vecr)}$ is a constant, that is to say $\phi(\vecr)$ has a constant curvature. 
\item The $B_m^2$ we defined depends entirely on material properties, hence it is called `material buckling.' For a reactor to be critical with $\keff = 1$, the material buckling is uniquely determined by the cross sections, 
\eqn{ B_m^2 = \frac{\nu \Sigma_f - \Sigma_a}{D} = \frac{\frac{\nu \Sigma_f}{\Sigma_a} - 1}{\frac{D}{\Sigma_a}} = \frac{\kinf - 1}{M^2} }
\item Solutions exist only for certain values of the buckling such that the flux is everywhere positive and vanishing on outer (or extrapolated) surfaces; we define these unique values as `geometrical buckling' $B_g^2$. The allowable values of geometrical buckling that satisfy the boundary conditions are uniquely determined reactor geometry.
\item For any reactor geometry, the reactor will be critical if and only if the materials satisfy:
  \eqn{ B_g^2 = B_m^2} 
Clearly there can be a solution when $\kinf > 1$. 
\end{enumerate}

\clearpage
\topic{Simple Geometry Laplacians}
In this section we are going to cover a couple of one group homogeneous geometry Laplacians. 
\begin{table}
  \centering
  \begin{tabular}{|c|c|} \hline
    Slab & $\dphidxn2 + B^2 \phi(x) = 0$ \\ \hline
    Sphere & $\dphidrn2 + \frac{2}{r} \dphidr + B^2 \phi(r) = 0$ \\ \hline
    Infinite Cylinder & $\dphidrn2 + \frac{1}{r} \dphidr + B^2 \phi(r) = 0$ \\ \hline
    Finite Cylinder & $\dphidrn2 + \frac{1}{r} \dphidr + \dphidzn2 + B^2 \phi(r,z) = 0$ \\ \hline
    Cartesian & $\dphidxn2 + \dphidyn2 + \dphidzn2 + B^2 \phi(x,y,z) = 0$ \\ \hline
  \end{tabular}
\end{table}




\clearpage
\topic{One Group Fundamental Mode Eigenvalues and Eigenvectors}
 Slab $\in  \left[- \frac{L}{2}, \frac{L}{2} \right]$:
  \eqn{ \dphidxn2 + B^2 \phi (x) &= 0, &\phi(x) &= A \cos (Bx) + C \sin(Bx) }
BCs: $\phi(\pm L/2) = 0$. Two equations two unknowns, 
  \eqn{ \left[ \begin{array}{cc} \cos(BL/2) & \sin(BL/2) \\ \cos(BL/2) & -\sin(BL/2) \end{array} \right] \left[ \begin{array}{cc} A \\ C \end{array} \right] = 0 }
  Set the determinant to be zero, we get $-2 \cos (BL/2) \sin (BL/2) = 0$. There are two possibilities: 
\eqn{ B_n&= \frac{n\pi}{L}  & \phi(x) &= \left\{ 
  \begin{array}{cc} 
    A_n \cos (B_n x) & n=1,3,5, \cdots \\
    A_n \sin (B_n x) & n=2,4,6, \cdots 
  \end{array} \right. }
But in order for $\phi(x) \ge 0$ everywhere, only $n=1$ is possible; that is, 
\eqn{ \phi(x) = A \cos \frac{\pi x}{L} }
and the criticality condition implies that, 
\eqn{ \frac{\nu \Sigma_f - \Sigma_a}{D}  = \left( \frac{\pi}{L} \right)^2 }


\begin{table}
  \centering
  \begin{tabular}{|l|l|l|} \hline
     Slab $\in  \left[- \frac{L}{2}, \frac{L}{2} \right]$ & $\phi(x) = A \cos \left( \frac{\pi x}{L} \right)$ & $B^2 = \left( \frac{\pi}{L} \right)^2$ \\ \hline
     Sphere $\in [0, R]$ & $\phi(r)= A\frac{\sin \left( \frac{\pi r}{R} \right)}{r} $ & $B^2 = \left( \frac{\pi}{R} \right)^2$ \\ \hline
     Infinite cylinder $\in [0, R]$ & $\phi(r) = A J_0 \left( \frac{2.405 r}{R} \right)$ & $B^2 = \left( \frac{2.405}{R} \right)^2$ \\ \hline
     Finite cylinder $r \in [0, R], z \in \left[ -\frac{H}{2}, \frac{H}{2} \right]$ & $\phi(r,z) = A J_0\left( \frac{2.405 r}{R} \right) \cos \left( \frac{\pi z}{H} \right)$ & $B^2 = \left( \frac{2.405}{R} \right)^2 + \left( \frac{\pi}{H} \right)^2$ \\ \hline
     Parallelepiped $\in \left[ -\frac{L_i}{2}, \frac{L_i}{2} \right]$ & $\phi(x) = A \cos \left( \frac{\pi x}{L_x} \right) \cos \left( \frac{\pi y}{L_y} \right) \cos \left( \frac{\pi x}{L_z} \right)$ & $B^2 = \left( \frac{\pi}{L_x} \right)^2 + \left( \frac{\pi}{L_y} \right)^2 + \left( \frac{\pi}{L_z} \right)^2$ \\ \hline
  \end{tabular}
\end{table}

Observations:
\begin{itemize}
\item The lowest node is the only one remains after the source is gone. 
\item For any positive value of materials buckling, there is a unique critical size for each reactor geometry. 
\item Example: know how to find height-to-diameter to minimize leakage. The optimal cylinder has a H/D of around 0.92. 
\end{itemize}


\clearpage
\topic{One Group Problem With Sources In A Subcritical Multiplying Medium}
Consider a subcritical multiplying medium, slab geometry from $-L/2$ to $L/2$, and with a source. The Helmholtz equation is,
\begin{align}
  - D \dphidxn2 + \Sigma_a \phi(x) &= \nu \Sigma_f \phi(x) + S(x) \\
  \dphidxn2 + B_m^2 \phi(x) &= -\frac{S(x)}{D} 
\end{align}
Because $B^2 = \frac{\nu \Sigma_f - \Sigma_a}{D} < 0 $ for a subcritical condition, we can re-write $B^2 = - |B|^2$, so our Helmholtz equation becomes,
\eqn{ \dphidxn2 - |B|^2 \phi(x) = S(x) }
The general solution is,
\eqn{ \phi_H (x) = A e^{|B|x} + Ce^{-|B|x} = A \cosh (|B|x) + C \sinh (|B| x) }
The particular solution depends on source $S(x)$. Apply BCs $\phi \left( \pm \frac{L}{2} \right) = 0$, we get the boundary conditions in the matrix form,
\eqn{ \left[ \begin{array}{cc} 
\cosh(|B|L/2) & \sinh(|B|L/2) \\
\sinh(|B|L/2) & \cosh(|B|L/2) 
\end{array} \right] 
\left[ \begin{array}{c}
A \\ C \end{array} \right] 
= - \left[ \begin{array}{c} 
\phi_p(L/2) \\ \phi_p(-L/2) \end{array} \right] }
Interpretations:
\begin{enumerate}
\item Coefficients A and C are uniquely determined for given source distribution; 
\item There always exists a physically realizable solution (no critical buckling!);
\item In the limit of $S\to 0$, the only physical solution is the trivial solution (\textcolor{red}{shouldn't the only solution be the general solution?}). 
\end{enumerate}





\end{document}
