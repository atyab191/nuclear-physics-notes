\documentclass{school-22.211-notes}
\date{March  7, 2012}

\begin{document}
\maketitle


%%%%%%%%%%%%%%%%%%%%%%%%% Resonance Models Day 6 %%%%%%%%%%%%%%%%%%%%%%%%%%%%
\clearpage
\topic{Observations From Two-Region PWR Pin Cell Applications}
(FIXME) update based on Lec10. 
\begin{enumerate}
\item With no oxygen or zirconium: flux dips in moderator are mild because there is only elastic scattering. 
\item Adding in Zr: Zr has massive elastic scattering resonances, hence creating fluctuations in both spectrums most noticable in energies right above the resonance energies. THe reactivity change is small (positive 560 pcm).
\item Adding in O: the fast flux looks distorded in both fuel and moderator, this is because the oxygen xs has werid dips in high energy xs. the reactivity change is not large (negative 680 pcm). 
\item Removing hydrogen absorption: the spectrum shape does not change much, but the reactivity change is huge (8808 pcm), because there are so much hydrogen in the system (hydrogen xs is 1/v down to 100 keV); this is why LWR fuel has to be enriched. 
\item Remove U238 fission: reduce reactivity by 2\% (negative -1910 pcm). Spectrum shape does not change. 
\item Doppler reactivity example (FIXME): see Lec 9. It is important to generate the Dopplar coefficient when preparing cross section data. 
\item Boron reactivity: Boron's capture cross section is almost entirely $(n,\alpha)$ cross section which produces He. Boron worth: $-16$ pcm/ppm. 
\end{enumerate}

Looking at the flux in the moderator divided by the flux in the fuel, we see that:
\begin{itemize}
\item Above resonance energies: fuel flux is depressed because absorption is high compared with fission; 
\item In resonance energies: larger than 1 because of fuel spectrums dips for the resonances; especially at the interval containing 6.7 eV;
\item Below resonance energies: the ratio is around 1, till we hit the really low energy (less than 0.1 eV), that there is a 1/E tail that the fuel flux gets depressed. 
\end{itemize}



\clearpage
\topic{Collision and Variance Tallies}
Starting from the infintie medium collision model, our two-region collision model is effectively an averaged model. For instance, $P_{ff}$ include cases like going from a fuel to moderator then back into fuel again, which is why we cannot use pathlength estimator. 

We almost always tally reaction rate or flux. Then at the end of the tallying, 
$\kinf$ and effective cross sections can be calculated from:
\eqn{ \kinf = \frac{\bar{\nu} \Sum \Sigma_f \phi}{\Sum \Sigma_a \phi} }
\eqn{ \bar{\Sigma}_{xg} = \frac{\Sum \Sigma_{xg} \phi_g}{\Sum \phi_g}}



\topic{HW4: Monte Carlo Pin Cell}
$k_{\infty}$: the number of absorption should equal to the total number of neutrons; in our case, we are losing a small portion of neutrons to the cutoff energy; we can cheat by pushing those neutrons back up. 

Flux disadvantage factor: (flux of the fuel times fuel volume) / (flux of the moderator times moderator volume). 

Our Monte Carlo tools predict LWR parameters reasonable accurately,
\begin{enumerate}
\item $\kinf$ has a bias of about -700pcm relative to CASMO-5;
\item Reactivity (differential) effects are very accurate:
  \begin{itemize}
  \item Enrichment worth is within a couple of percent;
  \item Doppler worth is within a couple of percent;
  \item Boron worth is within about 5 percent;
  \end{itemize}
\item We should be able to use this tool to generate accurate differential reactivity effects and cross section data for simple thermal reactor lattices;
\item Remaining errors are likely due to,
  \begin{itemize}
    \item Missing thermal bound scattering models;
    \item Missing inelastic scattering;
    \item Missing unresolved resonance models;
    \item Simplified 2-region treatment, isotropic scattering;
    \item $P_{ff}$ from rational approximation, simple Dancoff factor. 
  \end{itemize}
\end{enumerate}



\end{document}
