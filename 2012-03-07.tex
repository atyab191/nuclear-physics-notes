\documentclass{school-22.211-notes}
\date{March  7, 2012}

\begin{document}
\maketitle

\topic{k-infinity In Infinite Medium Calculations (No Leakage)}
Reference: Henry p. 104, Lec10 [FIXME] 
\eqn{ k_{\infty} =  \frac{\mbox{total fissions} \cdot \bar{\nu} }{\mbox{total absorptions}} = \frac{\int \dV \int \nu \overline{\Sigma_f} \Phi \dE}{\int \dV \int \overline{\Sigma_a} \Phi \dE} = \epsilon \eta f p }

\subtopic{k from one-group cross section}




\subtopic{k from two-group cross section}
\begin{enumerate}
\item General case: 
  \begin{align}
    \frac{\nu \overline{\Sigma_{f1}}}{\kinf} - \overline{\Sigma_{a1}} \phi_1 - \\
  \end{align}

\item Using effective removal cross section. We define an \hi{effective removal cross section} as,
  

   Although it looks like a chicken-to-egg problem that we include $\frac{\phi_2}{\phi_1}$ in $\hat{\Sigma}_{s12}$, my understanding is that we either know flux and needs to find cross section, or the other way around, hence we do not need to worry about this term. 

   If no upscattering, $\hat{\Sigma}_{s12}$ is effectively $\overline{\Sigma}_{s12}$, hence 
\end{enumerate}

\subtopic{k From Balance Equation}
For two region problem, we can also solve $\kinf$ from the neutron balance equation. 





The $\kinf$ from the three methods above should agree, because reaction rates conserves[FIXME].



\topic{Solving Two-Region Monte Carlo Spectral Analysis}
Review codes in Lec8. Review slide 22 on approximation of collision probabilities. 

\topic{Observations From Two-Region PWR Pin Cell Applications}
(FIXME) update based on Lec10. 
\begin{enumerate}
\item With no oxygen or zirconium: flux dips in moderator are mild because there is only elastic scattering. 
\item Adding in Zr: Zr has massive elastic scattering resonances, hence creating fluctuations in both spectrums most noticable in energies right above the resonance energies. THe reactivity change is small (positive 560 pcm).
\item Adding in O: the fast flux looks distorded in both fuel and moderator, this is because the oxygen xs has werid dips in high energy xs. the reactivity change is not large (negative 680 pcm). 
\item Removing hydrogen absorption: the spectrum shape does not change much, but the reactivity change is huge (8808 pcm), because there are so much hydrogen in the system (hydrogen xs is 1/v down to 100 keV); this is why LWR fuel has to be enriched. 
\item Remove U238 fission: reduce reactivity by 2\% (negative -1910 pcm). Spectrum shape does not change. 
\item Doppler reactivity example (FIXME): see Lec 9. It is important to generate the Dopplar coefficient when preparing cross section data. 
\item Boron reactivity (FIXME): Boron's capture cross section is almost entirely $(n,\alpha)$ cross section which produces He. 
\end{enumerate}

Looking at the flux in the moderator divided by the flux in the fuel, we see that:
\begin{itemize}
\item Above resonance energies: fuel flux is depressed because absorption is high compared with fission; 
\item In resonance energies: larger than 1 because of fuel spectrums dips for the resonances; especially at the interval containing 6.7 eV;
\item Below resonance energies: the ratio is around 1, till we hit the really low energy (less than 0.1 eV), that there is a 1/E tail that the fuel flux gets depressed. 
\end{itemize}

\topic{Collision and Variance Tallies}
Starting from the infintie medium collision model, our two-region collision model is effectively an averaged model. For instance, $P_{ff}$ include cases like going from a fuel to moderator then back into fuel again, which is why we cannot use pathlength estimator. 

We almost always tally reaction rate or flux. 
[FIXME] from lec 10.



\topic{HW4: Heterogeneous Spectral Calculations}
Cross section: provided by point-wise ENDF data, except we are going to overload the U238 resonance data for the lower energies.

Build bins: use 0.01 log(E) spacing (that's about 25,000 bins) to re-generate tables from the PENDF cross section. 

$k_{\infty}$: the number of absorption should equal to the total number of neutrons; in our case, we are losing a small portion of neutrons to the cutoff energy; we can cheat by pushing those neutrons back up. 

Flux disadvantage factor: (flux of the fuel times fuel volume) / (flux of the moderator times moderator volume). 

Generate statistics: do get an variance on the keff value, we keep a tally for each neutron (so accumulate the tally after each collision until the neutron is dead), and keep a copy of this tally and its square, and get the variance after all neutrons are done. 

Check the updated specs in Lec 10. 

\topic{Why We Need Deterministic Codes}
Lec10. Monte Carlo requires too much computing power. 


Pin-cell based, 


Assumbly based. 




In the next section we will discuss how to solve diffusion equation. 

\end{document}
