\documentclass{school-22.211-notes}
\date{February 27, 2012}

\begin{document}
\maketitle

\topic{Energy Self-Shielding Effects}
As we increase the U/H ratio, the RI decreases in the three big resonance regions, and we see big dips on the spectrum plot. When U/H = 1.0, the spectrum is distorted. There are so much U238 in the fuel, that there is no flux in the fuel anymore. As we increase the number of uranium atoms by a factor of 10, the number of absorption per atom is decreased by a factor of 3. That is, the total aborption still increases, but the absorption per atom decreases. 

Insert slide 26 here. 

\topic{Spectral Hardening}
\textit{As more absorber (uranium in this case) is added, the higher energy ranges become more important, as can be seen that the fraction of absorption happening in the higher energies increase, and the peak of the thermal spectrum shifts to the right as well. The 1/E shape becomes more important as well.}

Insert slide 28. 


\topic{Reactor Type Spectral Optimizations and Sensitivities}



\topic{Pset 3: Resonance Model, Effects of Self-shielding and Doppler}
Pset 3 adds U238 resonance model to our slowing down MC code. Notice only absorption resonance from 0 to 1 keV is considered for now. 

We should get column 1 and 4 in Lec 6,  slide 21. 


\end{document}
