\documentclass{school-22.211-notes}
\date{April  4, 2012}

\begin{document}
\maketitle

\lecture{Two-Group Diffusion: Analytical Solutions}
We are going to cover six classical examples to understand the mechanics, boundary conditions, and interface conditions. Know these for the exam. 
\topic{Point Source in Infinite Non-Multiplying Medium}


\topic{Plane Source in Infinite Non-Multiplying Medium}


\topic{Plane Source in Finite Multiplying Medium with $\kinf < 1$}




Notice in both finite and infinite case, the fluxes are convex; the finite curve is under the infinite curve. 



\topic{Plane Source in Finite Multiplying Medium with $\kinf > 1$}



If the reactor is critical, the flux at the source is is infinite. That is, there is no steady-state solution for flux at the source site. 



\topic{Critical Finite Cube}



\topic{Critical Finite Cylinder} 



\topic{Critical Reflected Slab Reactor}



\topic{Summary}
\begin{enumerate}
\item \textit{Super-positioning of sources}: if we are given a random source that is the sum of a couple of common forms of sources, we can super-position the flux from each of the common sources. This method works in any non-multiplying medium, and in multiplying medium in subcritical condition\footnote{Supercritical condition, as source increaes, flux inverts}. For instance, we can super-position point sources to get a line source. 
\item Diffusion theory is not valid near singularities. 
\item $\kinf$ is important\footnote{know this for the exams}: 
  \begin{enumerate}
  \item $\kinf = 1$: flux is straight line;
  \item $\kinf < 1$ means $B^2 < 1$, flux is $\sinh, \cosh$ which is convex; 
  \item $\kinf > 1$ means $B^2 > 0$, flux is $\sin, \cos$ which is concave. 
  \end{enumerate}
\item Know how to get the critical buckling for different geometries. 
\item Seperation of variables works as long as there is no one more than one direction that is heterogeneous. 
\end{enumerate}

  \begin{table}
    \centering
    \begin{tabular}{|c|c|c|} \hline
      Source & Geometry & Flux \\ \hline \hline
      Point & Infinite Slab & \\ \hline
    \end{tabular}
    \caption{Subcritical System: Source and Flux} 
  \end{table}
  
  
\end{document}
